% -*- coding: utf-8 -*-
% !TEX program = lualatex
\documentclass[oneside]{book}

% -*- coding: utf-8 -*-
% !TEX program = lualatex

\usepackage[a4paper,margin=2.5cm]{geometry}

\newcommand*{\myversion}{2022H}
\newcommand*{\mydate}{Version \myversion\ (\the\year-\mylpad\month-\mylpad\day)}
\newcommand*{\mylpad}[1]{\ifnum#1<10 0\the#1\else\the#1\fi}

\setlength{\parindent}{0pt}
\setlength{\parskip}{4pt plus 1pt minus 1pt}

\usepackage{codehigh}

\colorlet{highback}{blue9}
%\CodeHigh{lite}
\CodeHigh{language=latex/latex2,style/main=highback,style/code=highback}
\NewCodeHighEnv{code}{style/main=gray9,style/code=gray9}
\NewCodeHighEnv{demo}{style/main=gray9,style/code=gray9,demo}

\usepackage{enumitem}

\NewDocumentCommand\MySubScript{m}{$_{#1}$}

\ExplSyntaxOn
\NewDocumentCommand\PrintVarList{m}{
  \clist_set:Nn \l_tmpa_clist {#1}
  \clist_map_inline:Nn \l_tmpa_clist
    {
      \token_to_str:N ##1 ~
    }
}
\NewDocumentCommand\RelaceChacters{m}{
  \tl_set:Nn \lTmpaTl {#1}
  \regex_replace_once:nnN { \_ } { \c{MySubScript} } \lTmpaTl
}
\NewDocumentCommand\RelaceUnderScoreAll{m}{
  \tl_set:Nn \lTmpaTl {#1}
  \regex_replace_all:nnN { \_ } { \c{_} } \lTmpaTl
  \lTmpaTl
}
\ExplSyntaxOff

\NewDocumentEnvironment{variable}{om}{
  \vspace{5pt}
  \begin{minipage}{\linewidth}
  \hrule\vspace{4pt}\obeylines%
  \begingroup
  \ttfamily\bfseries\color{azure3}
  \PrintVarList{#2}
  \endgroup
  \par\vspace{4pt}\hrule
  \end{minipage}\par\nopagebreak\vspace{4pt}
}{%
  \vspace{5pt}%
}

\NewDocumentEnvironment{function}{om}{
  \vspace{5pt}%
}{\vspace{5pt}}

\NewDocumentEnvironment{syntax}{}{%
  \begin{minipage}{\linewidth}
  \hrule\vspace{4pt}\obeylines%
}{%
  \par\vspace{4pt}\hrule
  \end{minipage}\par\nopagebreak\vspace{4pt}
}

\NewDocumentEnvironment{texnote}{}{}{}

\NewDocumentCommand\cs{O{}m}{%
  \texttt{\bfseries\color{purple3}\cBackslashStr\RelaceUnderScoreAll{#2}}%
}
\NewDocumentCommand\meta{m}{%
  \RelaceChacters{#1}%
  \textsl{$\langle$\ignorespaces\lTmpaTl\unskip$\rangle$}%
}
\NewDocumentCommand\Arg{m}{%
  \RelaceChacters{#1}%
  \texttt{\{}\textsl{$\langle$\ignorespaces\lTmpaTl\unskip$\rangle$}\texttt{\}}%
}

\NewDocumentCommand\pkg{m}{\textsf{#1}}

\NewDocumentCommand\nan{}{\texttt{NaN}}
\NewDocumentCommand\enquote{m}{``#1''}

\let\tn=\cs

\RenewDocumentCommand\emph{m}{%
  \underline{\textsl{#1}}%
}

\newenvironment{l3regex-syntax}
  {\begin{itemize}\def\\{\char`\\}\def\makelabel##1{\hss\llap{\ttfamily##1}}}
  {\end{itemize}}

\usepackage{shortvrb}

\NewDocumentCommand \MyMakeShortVerb {} {%
  \MakeShortVerb \|%
  \MakeShortVerb \"%
}
\NewDocumentCommand \MyDeleteShortVerb {} {%
  \DeleteShortVerb \|%
  \DeleteShortVerb \"%
}

\AtBeginDocument{\MyMakeShortVerb}
\AtEndDocument{\MyDeleteShortVerb}
\AddToHook{env/demohigh/before}{\MyDeleteShortVerb}
\AddToHook{env/demohigh/after}{\MyMakeShortVerb}

\usepackage{hologo}
\providecommand*\LuaTeX{\hologo{LuaTeX}}
\providecommand*\pdfTeX{\hologo{pdfTeX}}
\providecommand*\XeTeX{\hologo{XeTeX}}

\usepackage{amsmath}

\usepackage{hyperref}
\hypersetup{
  colorlinks=true,
  urlcolor=blue3,
  linkcolor=blue3,
}

\usepackage{functional}
%\Functional{scoping=false,tracing=true}

\CodeHigh{lite}

\begin{document}

\chapter{Regular Expressions (\texttt{Regex})}

This module provides regular expression testing,
extraction of submatches, splitting, and replacement, all acting
on token lists. The syntax of regular expressions is mostly a subset
of the \textsc{pcre} syntax (and very close to \textsc{posix}),
with some additions
due to the fact that \TeX{} manipulates tokens rather than characters.
For performance reasons, only a limited set of features are implemented.
Notably, back-references are not supported.

Let us give a few examples. The following example replace the first
occurrence of \enquote{\texttt{at}} with \enquote{\texttt{is}}
in the token list variable \cs{lTmpaTl}.
\begin{demohigh}
\tlSet \lTmpaTl {That cat.}
\regexReplaceOnce {at} {is} \lTmpaTl
\tlUse \lTmpaTl
\end{demohigh}
A more complicated example is
a pattern to emphasize each word and add a comma after it:
\begin{demohigh}
\tlSet \lTmpaTl {That cat.}
\regexReplaceAll {\w+} {\c{underline} \cB\{ \0 \cE\} ,} \lTmpaTl
\tlUse \lTmpaTl
\end{demohigh}
The |\w| sequence represents any \enquote{word} character, and |+|
indicates that the |\w| sequence should be repeated as many times as
possible (at least once), hence matching a word in the input token
list. In the replacement text, |\0| denotes the full match (here, a
word).  The command |\underline| is inserted using |\c{underline}|,
and its argument |\0| is put between braces |\cB\{| and |\cE\}|.

If a regular expression is to be used several times,
it can be compiled once, and stored in a regex
variable using \cs{regexSet}. For example,
\begin{codehigh}
\regexNew \lFooRegex
\regexSet \lFooRegex {\c{begin} \cB. (\c[^BE].*) \cE.}
\end{codehigh}
stores in \cs{lFooRegex} a regular expression which matches the
starting marker for an environment: \cs[no-index]{begin}, followed by a
begin-group token (|\cB.|), then any number of tokens which are
neither begin-group nor end-group character tokens (|\c[^BE].*|),
ending with an end-group token (|\cE.|). As explained later,
the parentheses \enquote{capture} the result of |\c[^BE].*|,
giving us access to the name of the environment when doing
replacements.

\section{Regular Expression Variables}

If a regular expression is to be used several times,
it is better to compile it once rather than doing it
each time the regular expression is used. The compiled
regular expression is stored in a variable. All
of this module's functions can be given their
regular expression argument either as an explicit string
or as a compiled regular expression.

\begin{variable}{\lTmpaRegex,\lTmpbRegex,\lTmpcRegex,\lTmpiRegex,\lTmpjRegex,\lTmpkRegex}
Scratch regex variables for local assignment. These are never used by
\verb!function! package, and so are safe for use with any function.
However, they may be overwritten by other non-kernel
code and so should only be used for short-term storage.
\end{variable}

\begin{variable}{\gTmpaRegex,\gTmpbRegex,\gTmpcRegex,\gTmpiRegex,\gTmpjRegex,\gTmpkRegex}
Scratch regex variables for global assignment. These are never used by
\verb!function! package, and so are safe for use with any function.
However, they may be overwritten by other non-kernel
code and so should only be used for short-term storage.
\end{variable}

\begin{function}{\regexNew}
\begin{syntax}
\cs{regexNew} \meta{regex var}
\end{syntax}
Creates a new \meta{regex var} or raises an error if the
name is already taken. The declaration is global. The
\meta{regex var} is initially such that it never matches.
\end{function}

\begin{function}{\regexSet}
\begin{syntax}
\cs{regexSet} \meta{regex var} \Arg{regex}
\end{syntax}
Stores a compiled version of the \meta{regular expression} in the
\meta{regex var}. For instance, this function can be
used as
\begin{codehigh}
\regexNew \lMyRegex
\regexSet \lMyRegex {my\ (simple\ )? reg(ex|ular\ expression)}
\end{codehigh}
\end{function}

\begin{function}{\regexConst}
\begin{syntax}
\cs{regexConst} \meta{regex var} \Arg{regex}
\end{syntax}
Creates a new constant \meta{regex var} or raises an error if the name
is already taken.  The value of the \meta{regex var} is set
globally to the compiled version of the \meta{regular expression}.
\end{function}

\begin{function}{\regexLog,\regexVarLog,\regexShow,\regexVarShow}
\begin{syntax}
\cs{regexLog} \Arg{regex}
\cs{regexVarLog} \meta{regex var}
\cs{regexShow} \Arg{regex}
\cs{regexVarShow} \meta{regex var}
\end{syntax}
Displays in the terminal or writes in the log file (respectively)
how \pkg{l3regex} interprets the \meta{regex}. For instance,
\cs{regexShow} \verb+{\A X|Y}+ shows
\begin{codehigh}
+-branch
  anchor at start (\A)
  char code 88 (X)
+-branch
  char code 89 (Y)
\end{codehigh}
indicating that the anchor |\A| only applies to the first branch:
the second branch is not anchored to the beginning of the match.
\end{function}

\section{Regular Expression Matching}

\begin{function}{\regexMatch,\regexMatchT,\regexMatchF,\regexMatchTF}
\begin{syntax}
\cs{regexMatch} \Arg{regex} \Arg{token list}
\cs{regexMatchT} \Arg{regex} \Arg{token list} \Arg{true code}
\cs{regexMatchF} \Arg{regex} \Arg{token list} \Arg{false code}
\cs{regexMatchTF} \Arg{regex} \Arg{token list} \Arg{true code} \Arg{false code}
\end{syntax}
Tests whether the \meta{regular expression} matches any part
of the \meta{token list}. For instance,
\begin{demohigh}
\regexMatchTF {b [cde]*} {abecdcx} {\prgPrint{True}} {\prgPrint{False}}
\regexMatchTF {[b-dq-w]} {example} {\prgPrint{True}} {\prgPrint{False}}
\end{demohigh}
\end{function}

\begin{function}{\regexVarMatch,\regexVarMatchT,\regexVarMatchF,\regexVarMatchTF}
\begin{syntax}
\cs{regexVarMatch} \meta{regex var} \Arg{token list}
\cs{regexVarMatchT} \meta{regex var} \Arg{token list} \Arg{true code}
\cs{regexVarMatchF} \meta{regex var} \Arg{token list} \Arg{false code}
\cs{regexVarMatchTF} \meta{regex var} \Arg{token list} \Arg{true code} \Arg{false code}
\end{syntax}
Tests whether the \meta{regex var} matches any part of the \meta{token list}.
\end{function}

\begin{function}{\regexCount,\regexVarCount}
\begin{syntax}
\cs{regexCount} \Arg{regex} \Arg{token list} \meta{int var}
\cs{regexVarCount} \meta{regex var} \Arg{token list} \meta{int var}
\end{syntax}
Sets \meta{int var} within the current \TeX{} group level
equal to the number of times
\meta{regular expression} appears in \meta{token list}.
The search starts by finding the left-most longest match,
respecting greedy and lazy (non-greedy) operators. Then the search
starts again from the character following the last character
of the previous match, until reaching the end of the token list.
Infinite loops are prevented in the case where the regular expression
can match an empty token list: then we count one match between each
pair of characters. For instance,
\begin{demohigh}
\intNew \lFooInt
\regexCount {(b+|c)} {abbababcbb} \lFooInt
\intUse \lFooInt
\end{demohigh}
\end{function}

\begin{function}{\regexMatchCase}
\begin{syntax}
\cs{regexMatchCase}
~ ~ |{|
~ ~ ~ ~ \Arg{regex_1} \Arg{code case_1}
~ ~ ~ ~ \Arg{regex_2} \Arg{code case_2}
~ ~ ~ ~ \ldots
~ ~ ~ ~ \Arg{regex_n} \Arg{code case_n}
~ ~ |}| \Arg{token list}
\end{syntax}
Determines which of the \meta{regular expressions} matches at the earliest
point in the \meta{token list}, and leaves the corresponding \meta{code_i}.
If several \meta{regex} match starting at the same point,
then the first one in the list is selected and the others are discarded.
Each \meta{regex} can either be given as a regex variable or as an explicit
regular expression.
\par
In detail, for each starting position in the \meta{token list}, each
of the \meta{regex} is searched in turn.  If one of them matches
then the corresponding \meta{code} is used and everything else is
discarded, while if none of the \meta{regex} match at a given
position then the next starting position is attempted.  If none of
the \meta{regex} match anywhere in the \meta{token list} then
nothing is left in the input stream.  Note that this differs from
nested \cs{regexMatch} statements since all \meta{regex} are
attempted at each position rather than attempting to match
\meta{regex_1} at every position before moving on to \meta{regex_2}.
\end{function}

\begin{function}{\regexMatchCaseT}
\begin{syntax}
\cs{regexMatchCaseT}
~ ~ |{|
~ ~ ~ ~ \Arg{regex_1} \Arg{code case_1}
~ ~ ~ ~ \Arg{regex_2} \Arg{code case_2}
~ ~ ~ ~ \ldots
~ ~ ~ ~ \Arg{regex_n} \Arg{code case_n}
~ ~ |}| \Arg{token list}
~ ~ \Arg{true code}
\end{syntax}
Determines which of the \meta{regular expressions} matches at the
earliest point in the \meta{token list}, and leaves the
corresponding \meta{code_i} followed by the \meta{true code} in the
input stream. If several \meta{regex} match starting at the same
point, then the first one in the list is selected and the others are
discarded. Each \meta{regex} can either be given
as a regex variable or as an explicit regular expression.
\end{function}

\begin{function}{\regexMatchCaseF}
\begin{syntax}
\cs{regexMatchCaseF}
~ ~ |{|
~ ~ ~ ~ \Arg{regex_1} \Arg{code case_1}
~ ~ ~ ~ \Arg{regex_2} \Arg{code case_2}
~ ~ ~ ~ \ldots
~ ~ ~ ~ \Arg{regex_n} \Arg{code case_n}
~ ~ |}| \Arg{token list}
~ ~ \Arg{false code}
\end{syntax}
Determines which of the \meta{regular expressions} matches at the
earliest point in the \meta{token list}, and leaves the
corresponding \meta{code_i}.
If several \meta{regex} match starting at the same
point, then the first one in the list is selected and the others are
discarded.  If none of the \meta{regex} match, the \meta{false code}
is left in the input stream.  Each \meta{regex} can either be given
as a regex variable or as an explicit regular expression.
\end{function}

\begin{function}{\regexMatchCaseTF}
\begin{syntax}
\cs{regexMatchCaseTF}
~ ~ |{|
~ ~ ~ ~ \Arg{regex_1} \Arg{code case_1}
~ ~ ~ ~ \Arg{regex_2} \Arg{code case_2}
~ ~ ~ ~ \ldots
~ ~ ~ ~ \Arg{regex_n} \Arg{code case_n}
~ ~ |}| \Arg{token list}
~ ~ \Arg{true code} \Arg{false code}
\end{syntax}
Determines which of the \meta{regular expressions} matches at the
earliest point in the \meta{token list}, and leaves the
corresponding \meta{code_i} followed by the \meta{true code} in the
input stream.  If several \meta{regex} match starting at the same
point, then the first one in the list is selected and the others are
discarded.  If none of the \meta{regex} match, the \meta{false code}
is left in the input stream.  Each \meta{regex} can either be given
as a regex variable or as an explicit regular expression.
\end{function}

\section{Regular Expression Submatch Extraction}

\begin{function}{\regexExtractOnce,\regexExtractOnceT,\regexExtractOnceF,\regexExtractOnceTF}
\begin{syntax}
\cs{regexExtractOnce} \Arg{regex} \Arg{token list} \meta{seq var}
\cs{regexExtractOnceT} \Arg{regex} \Arg{token list} \meta{seq var} \Arg{true code}
\cs{regexExtractOnceF} \Arg{regex} \Arg{token list} \meta{seq var} \Arg{false code}
\cs{regexExtractOnceTF} \Arg{regex} \Arg{token list} \meta{seq var} \Arg{true code} \Arg{false code}
\end{syntax}
Finds the first match of the \meta{regular expression} in the
\meta{token list}. If it exists, the match is stored as the first
item of the \meta{seq var}, and further items are the contents of
capturing groups, in the order of their opening parenthesis. The
\meta{seq var} is assigned locally. If there is no match, the
\meta{seq var} is cleared.  The testing versions insert the
\meta{true code} into the input stream if a match was found, and the
\meta{false code} otherwise.
\par
For instance, assume that you type
\begin{codehigh}
\regexExtractOnce {\A(La)?TeX(!*)\Z} {LaTeX!!!} \lTmpaSeq
\end{codehigh}
Then the regular expression (anchored at the start with |\A| and
at the end with |\Z|) must match the whole token list. The first
capturing group, |(La)?|, matches |La|, and the second capturing
group, |(!*)|, matches |!!!|. Thus, \cs{lTmpaSeq} contains as a result
the items |{LaTeX!!!}|, |{La}|, and |{!!!}|.
Note that the $n$-th item of \cs{lTmpaSeq}, as obtained using
\cs{seqVarItem}, correspond to the submatch numbered $(n-1)$ in
functions such as \cs{regexReplaceOnce}.
\end{function}

\begin{function}{\regexVarExtractOnce,\regexVarExtractOnceT,\regexVarExtractOnceF,\regexVarExtractOnceTF}
\begin{syntax}
\cs{regexVarExtractOnce} \meta{regex var} \Arg{token list} \meta{seq var}
\cs{regexVarExtractOnceT} \meta{regex var} \Arg{token list} \meta{seq var} \Arg{true code}
\cs{regexVarExtractOnceF} \meta{regex var} \Arg{token list} \meta{seq var} \Arg{false code}
\cs{regexVarExtractOnceTF} \meta{regex var} \Arg{token list} \meta{seq var} \Arg{true code} \Arg{false code}
\end{syntax}
Finds the first match of the \meta{regex var} in the
\meta{token list}. If it exists, the match is stored as the first
item of the \meta{seq var}, and further items are the contents of
capturing groups, in the order of their opening parenthesis. The
\meta{seq var} is assigned locally. If there is no match, the
\meta{seq var} is cleared.  The testing versions insert the
\meta{true code} into the input stream if a match was found, and the
\meta{false code} otherwise.
\end{function}

\begin{function}{\regexExtractAll,\regexExtractAllT,\regexExtractAllF,\regexExtractAllTF}
\begin{syntax}
\cs{regexExtractAll} \Arg{regex} \Arg{token list} \meta{seq var}
\cs{regexExtractAllT} \Arg{regex} \Arg{token list} \meta{seq var} \Arg{true code}
\cs{regexExtractAllF} \Arg{regex} \Arg{token list} \meta{seq var} \Arg{false code}
\cs{regexExtractAllTF} \Arg{regex} \Arg{token list} \meta{seq var} \Arg{true code} \Arg{false code}
\end{syntax}
Finds all matches of the \meta{regular expression}
in the \meta{token list}, and stores all the submatch information
in a single sequence (concatenating the results of
multiple \cs{regexExtractOnce} calls).
The \meta{seq var} is assigned locally. If there is no match,
the \meta{seq var} is cleared.
The testing versions insert the \meta{true code} into the input
stream if a match was found, and the \meta{false code} otherwise.
For instance, assume that you type
\begin{codehigh}
\regexExtractAll {\w+} {Hello, world!} \lTmpaSeq
\end{codehigh}
Then the regular expression matches twice, the resulting
sequence contains the two items |{Hello}| and |{world}|.
\end{function}

\begin{function}{\regexVarExtractAll,\regexVarExtractAllT,\regexVarExtractAllF,\regexVarExtractAllTF}
\begin{syntax}
\cs{regexVarExtractAll} \meta{regex var} \Arg{token list} \meta{seq var}
\cs{regexVarExtractAllT} \meta{regex var} \Arg{token list} \meta{seq var} \Arg{true code}
\cs{regexVarExtractAllF} \meta{regex var} \Arg{token list} \meta{seq var} \Arg{false code}
\cs{regexVarExtractAllTF} \meta{regex var} \Arg{token list} \meta{seq var} \Arg{true code} \Arg{false code}
\end{syntax}
Finds all matches of the \meta{regex var}
in the \meta{token list}, and stores all the submatch information
in a single sequence (concatenating the results of
multiple \cs{regexVarExtractOnce} calls).
The \meta{seq var} is assigned locally. If there is no match,
the \meta{seq var} is cleared.
The testing versions insert the \meta{true code} into the input
stream if a match was found, and the \meta{false code} otherwise.
\end{function}

\begin{function}{\regexSplit,\regexSplitT,\regexSplitF,\regexSplitTF}
\begin{syntax}
\cs{regexSplit} \Arg{regular expression} \Arg{token list} \meta{seq var}
\cs{regexSplitT} \Arg{regular expression} \Arg{token list} \meta{seq var} \Arg{true code}
\cs{regexSplitF} \Arg{regular expression} \Arg{token list} \meta{seq var} \Arg{false code}
\cs{regexSplitTF} \Arg{regular expression} \Arg{token list} \meta{seq var} \Arg{true code} \Arg{false code}
\end{syntax}
Splits the \meta{token list} into a sequence of parts, delimited by
matches of the \meta{regular expression}. If the \meta{regular expression}
has capturing groups, then the token lists that they match are stored as
items of the sequence as well. The assignment to \meta{seq var} is local.
If no match is found the resulting \meta{seq var} has the
\meta{token list} as its sole item. If the \meta{regular expression}
matches the empty token list, then the \meta{token list} is split
into single tokens.
The testing versions insert the \meta{true code} into the input
stream if a match was found, and the \meta{false code} otherwise.
For example, after
\begin{codehigh}
\seqNew \lPathSeq
\regexSplit {/} {the/path/for/this/file.tex} \lPathSeq
\end{codehigh}
the sequence |\lPathSeq| contains the items |{the}|, |{path}|,
|{for}|, |{this}|, and |{file.tex}|.
\end{function}

\begin{function}{\regexVarSplit,\regexVarSplitT,\regexVarSplitF,\regexVarSplitTF}
\begin{syntax}
\cs{regexVarSplit} \meta{regex var} \Arg{token list} \meta{seq var}
\cs{regexVarSplitT} \meta{regex var} \Arg{token list} \meta{seq var} \Arg{true code}
\cs{regexVarSplitF} \meta{regex var} \Arg{token list} \meta{seq var} \Arg{false code}
\cs{regexVarSplitTF} \meta{regex var} \Arg{token list} \meta{seq var} \Arg{true code} \Arg{false code}
\end{syntax}
Splits the \meta{token list} into a sequence of parts, delimited by
matches of the \meta{regular expression}. If the \meta{regex var}
has capturing groups, then the token lists that they match are stored as
items of the sequence as well. The assignment to \meta{seq var} is local.
If no match is found the resulting \meta{seq var} has the
\meta{token list} as its sole item. If the \meta{regular expression}
matches the empty token list, then the \meta{token list} is split
into single tokens.
The testing versions insert the \meta{true code} into the input
stream if a match was found, and the \meta{false code} otherwise.
\end{function}

\section{Regular Expression Replacement}

\begin{function}{\regexReplaceOnce,\regexReplaceOnceT,\regexReplaceOnceT,\regexReplaceOnceTF}
\begin{syntax}
\cs{regexReplaceOnce} \Arg{regular expression} \Arg{replacement} \meta{tl var}
\cs{regexReplaceOnceT} \Arg{regular expression} \Arg{replacement} \meta{tl var} \Arg{true code}
\cs{regexReplaceOnceF} \Arg{regular expression} \Arg{replacement} \meta{tl var} \Arg{false code}
\cs{regexReplaceOnceTF} \Arg{regular expression} \Arg{replacement} \meta{tl var} \Arg{true code} \Arg{false code}
\end{syntax}
Searches for the \meta{regular expression} in the contents of the
\meta{tl var} and replaces the first match with the
\meta{replacement}. In the \meta{replacement},
|\0| represents the full match, |\1| represent the contents of the
first capturing group, |\2| of the second, \emph{etc.}
The result is assigned locally to \meta{tl var}.
\end{function}

\begin{function}{\regexReplaceOnce,\regexReplaceOnceT,\regexReplaceOnceT,\regexReplaceOnceTF}
\begin{syntax}
\cs{regexVarReplaceOnce} \meta{regex var} \Arg{replacement} \meta{tl var}
\cs{regexVarReplaceOnceT} \meta{regex var} \Arg{replacement} \meta{tl var} \Arg{true code}
\cs{regexVarReplaceOnceF} \meta{regex var} \Arg{replacement} \meta{tl var} \Arg{false code}
\cs{regexVarReplaceOnceTF} \meta{regex var} \Arg{replacement} \meta{tl var} \Arg{true code} \Arg{false code}
\end{syntax}
Searches for the \meta{regex var} in the contents of the
\meta{tl var} and replaces the first match with the
\meta{replacement}. In the \meta{replacement},
|\0| represents the full match, |\1| represent the contents of the
first capturing group, |\2| of the second, \emph{etc.}
The result is assigned locally to \meta{tl var}.
\end{function}

\begin{function}{\regexReplaceAll,\regexReplaceAllT,\regexReplaceAllF,\regexReplaceAllTF}
\begin{syntax}
\cs{regexReplaceAll} \Arg{regular expression} \Arg{replacement} \meta{tl var}
\cs{regexReplaceAllT} \Arg{regular expression} \Arg{replacement} \meta{tl var} \Arg{true code}
\cs{regexReplaceAllF} \Arg{regular expression} \Arg{replacement} \meta{tl var} \Arg{false code}
\cs{regexReplaceAllTF} \Arg{regular expression} \Arg{replacement} \meta{tl var} \Arg{true code} \Arg{false code}
\end{syntax}
Replaces all occurrences of the \meta{regex var} in the
contents of the \meta{tl var}
by the \meta{replacement}, where |\0| represents
the full match, |\1| represent the contents of the first capturing
group, |\2| of the second, \emph{etc.} Every match is treated
independently, and matches cannot overlap.  The result is assigned
locally to \meta{tl~var}.
\end{function}

\begin{function}{\regexVarReplaceAll,\regexVarReplaceAllT,\regexVarReplaceAllF,\regexVarReplaceAllTF}
\begin{syntax}
\cs{regexVarReplaceAll} \meta{regex var} \Arg{replacement} \meta{tl var}
\cs{regexVarReplaceAllT} \meta{regex var} \Arg{replacement} \meta{tl var} \Arg{true code}
\cs{regexVarReplaceAllF} \meta{regex var} \Arg{replacement} \meta{tl var} \Arg{false code}
\cs{regexVarReplaceAllTF} \meta{regex var} \Arg{replacement} \meta{tl var} \Arg{true code} \Arg{false code}
\end{syntax}
Replaces all occurrences of the \meta{regular expression} in the
contents of the \meta{tl var}
by the \meta{replacement}, where |\0| represents
the full match, |\1| represent the contents of the first capturing
group, |\2| of the second, \emph{etc.} Every match is treated
independently, and matches cannot overlap.  The result is assigned
locally to \meta{tl var}.
\end{function}

\begin{function}{\regexReplaceCaseOnce}
\begin{syntax}
\cs{regexReplaceCaseOnce}
~ ~ |{|
~ ~ ~ ~ \Arg{regex_1} \Arg{replacement_1}
~ ~ ~ ~ \Arg{regex_2} \Arg{replacement_2}
~ ~ ~ ~ \ldots
~ ~ ~ ~ \Arg{regex_n} \Arg{replacement_n}
~ ~ |}| \meta{tl var}
\end{syntax}
Replaces the earliest match of the regular expression
"(?|"\meta{regex_1}"|"\dots"|"\meta{regex_n}")" in the \meta{token list variable}
by the \meta{replacement} corresponding to which \meta{regex_i} matched.
If none of the \meta{regex} match, then the
\meta{tl var} is not modified. Each \meta{regex} can either be given as a regex
variable or as an explicit regular expression.
\par
In detail, for each starting position in the \meta{token list}, each
of the \meta{regex} is searched in turn.  If one of them matches
then it is replaced by the corresponding \meta{replacement} as
described for \cs{regexReplaceOnce}.  This is equivalent to
checking with \cs{regexMatchCase} which \meta{regex} matches,
then performing the replacement with \cs{regexReplaceOnce}.
\end{function}

\begin{function}{\regexReplaceCaseOnceT}
\begin{syntax}
\cs{regexReplaceCaseOnceT}
~ ~ |{|
~ ~ ~ ~ \Arg{regex_1} \Arg{replacement_1}
~ ~ ~ ~ \Arg{regex_2} \Arg{replacement_2}
~ ~ ~ ~ \ldots
~ ~ ~ ~ \Arg{regex_n} \Arg{replacement_n}
~ ~ |}| \meta{tl var}
~ ~ \Arg{true code}
\end{syntax}
Replaces the earliest match of the regular expression
"(?|"\meta{regex_1}"|"\dots"|"\meta{regex_n}")" in the \meta{token list variable}
by the \meta{replacement} corresponding to which
\meta{regex_i} matched, then leaves the \meta{true code} in the
input stream. If none of the \meta{regex} match, then the
\meta{tl var} is not modified. Each \meta{regex} can either be given as a regex
variable or as an explicit regular expression.
\end{function}

\begin{function}{\regexReplaceCaseOnceF}
\begin{syntax}
\cs{regexReplaceCaseOnceF}
~ ~ |{|
~ ~ ~ ~ \Arg{regex_1} \Arg{replacement_1}
~ ~ ~ ~ \Arg{regex_2} \Arg{replacement_2}
~ ~ ~ ~ \ldots
~ ~ ~ ~ \Arg{regex_n} \Arg{replacement_n}
~ ~ |}| \meta{tl var}
~ ~ \Arg{false code}
\end{syntax}
Replaces the earliest match of the regular expression
"(?|"\meta{regex_1}"|"\dots"|"\meta{regex_n}")" in the \meta{token list variable}
by the \meta{replacement} corresponding to which
\meta{regex_i} matched. If none of the \meta{regex} match, then the
\meta{tl var} is not modified, and the \meta{false code} is left in
the input stream.  Each \meta{regex} can either be given as a regex
variable or as an explicit regular expression.
\end{function}

\begin{function}{\regexReplaceCaseOnceTF}
\begin{syntax}
\cs{regexReplaceCaseOnceTF}
~ ~ |{|
~ ~ ~ ~ \Arg{regex_1} \Arg{replacement_1}
~ ~ ~ ~ \Arg{regex_2} \Arg{replacement_2}
~ ~ ~ ~ \ldots
~ ~ ~ ~ \Arg{regex_n} \Arg{replacement_n}
~ ~ |}| \meta{tl var}
~ ~ \Arg{true code} \Arg{false code}
\end{syntax}
Replaces the earliest match of the regular expression
"(?|"\meta{regex_1}"|"\dots"|"\meta{regex_n}")" in the \meta{token list variable}
by the \meta{replacement} corresponding to which
\meta{regex_i} matched, then leaves the \meta{true code} in the
input stream.  If none of the \meta{regex} match, then the
\meta{tl var} is not modified, and the \meta{false code} is left in
the input stream.  Each \meta{regex} can either be given as a regex
variable or as an explicit regular expression.
\end{function}

\begin{function}{\regexReplaceCaseAll}
\begin{syntax}
\cs{regexReplaceCaseAll}
~ ~ |{|
~ ~ ~ ~ \Arg{regex_1} \Arg{replacement_1}
~ ~ ~ ~ \Arg{regex_2} \Arg{replacement_2}
~ ~ ~ ~ \ldots
~ ~ ~ ~ \Arg{regex_n} \Arg{replacement_n}
~ ~ |}| \meta{tl var}
\end{syntax}
Replaces all occurrences of all \meta{regex} in the \meta{token~list}
by the corresponding \meta{replacement}. Every match is
treated independently, and matches cannot overlap. The result is
assigned locally to \meta{tl var}.
\par
In detail, for each starting position in the \meta{token list}, each
of the \meta{regex} is searched in turn.  If one of them matches
then it is replaced by the corresponding \meta{replacement}, and the
search resumes at the position that follows this match (and
replacement).  For instance
%% FIXME
%\begin{codehigh}
%\tlSet \lTmpaTl {Hello, world!}
%\regexReplaceCaseAll
%  {
%    {[A-Za-z]+} {``\0''}
%    {\b}        {---}
%    {.}         {[\0]}
%  } \lTmpaTl
%\end{codehigh}
\begin{codehigh}
\tlSet \lTmpaTl {Hello, world!}
\regexReplaceCaseAll
  {
    {[A-Za-z]+} {``\0''}
    {\b} {---}
    {.} {[\0]}
  } \lTmpaTl
\end{codehigh}
results in \cs{lTmpaTl} having the contents
\verb*|``Hello''---[,][ ]``world''---[!]|.  Note in particular that
the word-boundary assertion |\b| did not match at the start of words
because the case |[A-Za-z]+| matched at these positions.  To change
this, one could simply swap the order of the two cases in the
argument of \cs{regexReplaceCaseAll}.
\end{function}

\begin{function}{\regexReplaceCaseAllT}
\begin{syntax}
\cs{regexReplaceCaseAllT}
~ ~ |{|
~ ~ ~ ~ \Arg{regex_1} \Arg{replacement_1}
~ ~ ~ ~ \Arg{regex_2} \Arg{replacement_2}
~ ~ ~ ~ \ldots
~ ~ ~ ~ \Arg{regex_n} \Arg{replacement_n}
~ ~ |}| \meta{tl var}
~ ~ \Arg{true code}
\end{syntax}
Replaces all occurrences of all \meta{regex} in the \meta{token~list}
by the corresponding \meta{replacement}.  Every match is
treated independently, and matches cannot overlap. The result is
assigned locally to \meta{tl var}, and the \meta{true code}
is left in the input stream if any replacement was made.
\end{function}

\begin{function}{\regexReplaceCaseAllF}
\begin{syntax}
\cs{regexReplaceCaseAllF}
~ ~ |{|
~ ~ ~ ~ \Arg{regex_1} \Arg{replacement_1}
~ ~ ~ ~ \Arg{regex_2} \Arg{replacement_2}
~ ~ ~ ~ \ldots
~ ~ ~ ~ \Arg{regex_n} \Arg{replacement_n}
~ ~ |}| \meta{tl var}
~ ~ \Arg{false code}
\end{syntax}
Replaces all occurrences of all \meta{regex} in the \meta{token~list}
by the corresponding \meta{replacement}.  Every match is
treated independently, and matches cannot overlap. The result is
assigned locally to \meta{tl var}, and the \meta{false code} is left
in the input stream if not any replacement was made.
\end{function}

\begin{function}{\regexReplaceCaseAllTF}
\begin{syntax}
\cs{regexReplaceCaseAllTF}
~ ~ |{|
~ ~ ~ ~ \Arg{regex_1} \Arg{replacement_1}
~ ~ ~ ~ \Arg{regex_2} \Arg{replacement_2}
~ ~ ~ ~ \ldots
~ ~ ~ ~ \Arg{regex_n} \Arg{replacement_n}
~ ~ |}| \meta{tl var}
~ ~ \Arg{true code} \Arg{false code}
\end{syntax}
Replaces all occurrences of all \meta{regex} in the \meta{token
list} by the corresponding \meta{replacement}.  Every match is
treated independently, and matches cannot overlap.  The result is
assigned locally to \meta{tl var}, and the \meta{true code} or
\meta{false code} is left in the input stream depending on whether
any replacement was made or not.
\end{function}

\section{Syntax of Regular Expressions}

\subsection{Regular Expression Examples}

We start with a few examples, and encourage the reader to apply
\cs{regexShow} to these regular expressions.
\begin{itemize}
\item |Cat| matches the word \enquote{Cat} capitalized in this way,
  but also matches the beginning of the word \enquote{Cattle}: use
  |\bCat\b| to match a complete word only.
\item |[abc]| matches one letter among \enquote{a}, \enquote{b},
  \enquote{c}; the pattern \verb"(a|b|c)" matches the same three
  possible letters (but see the discussion of submatches below).
\item |[A-Za-z]*| matches any number (due to the quantifier
  \verb"*") of Latin letters (not accented).
\item |\c{[A-Za-z]*}| matches a control sequence made of Latin
  letters.
\item |\_[^\_]*\_| matches an underscore, any number of characters
  other than underscore, and another underscore; it is equivalent to
  |\_.*?\_| where |.| matches arbitrary characters and the
  lazy quantifier |*?| means to match as few characters as
  possible, thus avoiding matching underscores.
\item |[\+\-]?\d+| matches an explicit integer with at most one
  sign.
\item \verb*"[\+\-\ ]*\d+\ *" matches an explicit integer with any
  number of $+$ and $-$ signs, with spaces allowed except within the
  mantissa, and surrounded by spaces.
\item \verb*"[\+\-\ ]*(\d+|\d*\.\d+)\ *" matches an explicit integer or
  decimal number; using \verb*"[.,]" instead of \verb*"\." would allow
  the comma as a decimal marker.
\item
  \verb*"[\+\-\ ]*(\d+|\d*\.\d+)\ *((?i)pt|in|[cem]m|ex|[bs]p|[dn]d|[pcn]c)\ *"
  \allowbreak matches an explicit dimension with any unit that \TeX{} knows, where
  \verb*"(?i)" means to treat lowercase and uppercase letters
  identically.
\item \verb*"[\+\-\ ]*((?i)nan|inf|(\d+|\d*\.\d+)(\ *e[\+\-\ ]*\d+)?)\ *"
  matches an explicit floating point number or the special values
  \verb*"nan" and \verb*"inf" (with signs and spaces allowed).
\item \verb*"[\+\-\ ]*(\d+|\cC.)\ *" matches an explicit integer or
  control sequence (without checking whether it is an integer
  variable).
\item |\G.*?\K| at the beginning of a regular expression matches and
  discards (due to |\K|) everything between the end of the previous
  match (|\G|) and what is matched by the rest of the regular
  expression; this is useful in \cs{regexReplaceAll} when the
  goal is to extract matches or submatches in a finer way than with
  \cs{regexExtractAll}.
\end{itemize}
While it is impossible for a regular expression to match only integer
expressions, \newline\verb*"[\+\-\(]*\d+\)*([\+\-*/][\+\-\(]*\d+\)*)*" matches among
other things all valid integer expressions (made only with explicit
integers).  One should follow it with further testing.

\subsection{Characters in Regular Expressions}

Most characters match exactly themselves,
with an arbitrary category code. Some characters are
special and must be escaped with a backslash (\emph{e.g.}, |\*|
matches a star character). Some escape sequences of
the form backslash--letter also have a special meaning
(for instance |\d| matches any digit). As a rule,
\begin{itemize}
\item every alphanumeric character (\texttt{A}--\texttt{Z},
  \texttt{a}--\texttt{z}, \texttt{0}--\texttt{9}) matches
  exactly itself, and should not be escaped, because
  |\A|, |\B|, \ldots{} have special meanings;
\item non-alphanumeric printable ASCII characters can (and should)
  always be escaped: many of them have special meanings (\emph{e.g.},
  use |\(|, |\)|, |\?|, |\.|, |\^|);
\item spaces should always be escaped (even in character
  classes);
\item any other character may be escaped or not, without any
  effect: both versions match exactly that character.
\end{itemize}
Note that these rules play nicely with the fact that many
non-alphanumeric characters are difficult to input into \TeX{}
under normal category codes. For instance, |\\abc\%|
matches the characters |\abc%| (with arbitrary category codes),
but does not match the control sequence |\abc| followed by a
percent character. Matching control sequences can be done
using the |\c|\Arg{regex} syntax (see below).

Any special character which appears at a place where its special
behaviour cannot apply matches itself instead (for instance, a
quantifier appearing at the beginning of a string), after raising a
warning.

Characters.
\begin{l3regex-syntax}
  \item[\\x\{hh\ldots{}\}] Character with hex code \texttt{hh\ldots{}}
  \item[\\xhh] Character with hex code \texttt{hh}.
  \item[\\a] Alarm (hex 07).
  \item[\\e] Escape (hex 1B).
  \item[\\f] Form-feed (hex 0C).
  \item[\\n] New line (hex 0A).
  \item[\\r] Carriage return (hex 0D).
  \item[\\t] Horizontal tab (hex 09).
\end{l3regex-syntax}

\subsection{Characters Classes}

Character types.
\begin{l3regex-syntax}
  \item[.] A single period matches any token.
  \item[\\d] Any decimal digit.
  \item[\\h] Any horizontal space character,
    equivalent to |[\ \^^I]|: space and tab.
  \item[\\s] Any space character,
    equivalent to |[\ \^^I\^^J\^^L\^^M]|.
  \item[\\v] Any vertical space character,
    equivalent to |[\^^J\^^K\^^L\^^M]|. Note that |\^^K| is a vertical space,
    but not a space, for compatibility with Perl.
  \item[\\w] Any word character, \emph{i.e.},
    alphanumerics and underscore, equivalent to the explicit
    class |[A-Za-z0-9\_]|.
  \item[\\D] Any token not matched by |\d|.
  \item[\\H] Any token not matched by |\h|.
  \item[\\N] Any token other than the |\n| character (hex 0A).
  \item[\\S] Any token not matched by |\s|.
  \item[\\V] Any token not matched by |\v|.
  \item[\\W] Any token not matched by |\w|.
\end{l3regex-syntax}
Of those, |.|, |\D|, |\H|, |\N|, |\S|, |\V|, and |\W| match arbitrary
control sequences.

Character classes match exactly one token in the subject.
\begin{l3regex-syntax}
  \item[{[\ldots{}]}] Positive character class.
    Matches any of the specified tokens.
  \item[{[\char`\^\ldots{}]}] Negative character class.
    Matches any token other than the specified characters.
  \item[{x-y}] Within a character class, this denotes a range (can be
    used with escaped characters).
  \item[{[:\meta{name}:]}] Within a character class (one more set of
    brackets), this denotes the \textsc{posix} character class
    \meta{name}, which can be \texttt{alnum}, \texttt{alpha},
    \texttt{ascii}, \texttt{blank}, \texttt{cntrl}, \texttt{digit},
    \texttt{graph}, \texttt{lower}, \texttt{print}, \texttt{punct},
    \texttt{space}, \texttt{upper}, \texttt{word}, or \texttt{xdigit}.
  \item[{[:\char`\^\meta{name}:]}] Negative \textsc{posix} character class.
\end{l3regex-syntax}
For instance, |[a-oq-z\cC.]| matches any lowercase latin letter
except |p|, as well as control sequences (see below for a description
of |\c|).

In character classes, only |[|, |^|, |-|, |]|, |\| and spaces are
special, and should be escaped. Other non-alphanumeric characters can
still be escaped without harm. Any escape sequence which matches a
single character (|\d|, |\D|, \emph{etc.}) is supported in character
classes.  If the first character is |^|, then
the meaning of the character class is inverted; |^| appearing anywhere
else in the range is not special.  If the first character (possibly
following a leading |^|) is |]| then it does not need to be escaped
since ending the range there would make it empty.
Ranges of characters
can be expressed using |-|, for instance, |[\D 0-5]| and |[^6-9]| are
equivalent.

\subsection{Structure: Alternatives, Groups, Repetitions}

Quantifiers (repetition).
\begin{l3regex-syntax}
  \item[?] $0$ or $1$, greedy.
  \item[??] $0$ or $1$, lazy.
  \item[*] $0$ or more, greedy.
  \item[*?] $0$ or more, lazy.
  \item[+] $1$ or more, greedy.
  \item[+?] $1$ or more, lazy.
  \item[\{$n$\}] Exactly $n$.
  \item[\{$n,$\}] $n$ or more, greedy.
  \item[\{$n,$\}?] $n$ or more, lazy.
  \item[\{$n,m$\}] At least $n$, no more than $m$, greedy.
  \item[\{$n,m$\}?] At least $n$, no more than $m$, lazy.
\end{l3regex-syntax}
For greedy quantifiers the regex code will first investigate matches
that involve as many repetitions as possible, while for lazy
quantifiers it investigates matches with as few repetitions as
possible first.

Alternation and capturing groups.
\begin{l3regex-syntax}
  \item[A\char`|B\char`|C] Either one of \texttt{A}, \texttt{B},
    or \texttt{C}, investigating \texttt{A} first.
  \item[(\ldots{})] Capturing group.
  \item[(?:\ldots{})] Non-capturing group.
  \item[(?\char`|\ldots{})] Non-capturing group which resets
    the group number for capturing groups in each alternative.
    The following group is numbered with the first unused
    group number.
\end{l3regex-syntax}

Capturing groups are a means of extracting information about the
match. Parenthesized groups are labelled in the order of their
opening parenthesis, starting at $1$. The contents of those groups
corresponding to the \enquote{best} match (leftmost longest)
can be extracted and stored in a sequence of token lists using for
instance \cs{regexExtractOnceTF}.

The |\K| escape sequence resets the beginning of the match to the
current position in the token list. This only affects what is reported
as the full match. For instance,
\begin{codehigh}
\regexExtractAll {a \K .} {a123aaxyz} \lFooSeq
\end{codehigh}
results in \cs{lFooSeq} containing the items |{1}| and |{a}|: the
true matches are |{a1}| and |{aa}|, but they are trimmed by the use of
|\K|. The |\K| command does not affect capturing groups: for instance,
\begin{codehigh}
\regexExtractOnce {(. \K c)+ \d} {acbc3} \lFooSeq
\end{codehigh}
results in \cs{lFooSeq} containing the items |{c3}| and |{bc}|: the
true match is |{acbc3}|, with first submatch |{bc}|, but |\K| resets
the beginning of the match to the last position where it appears.

\subsection{Matching Exact Tokens}

The |\c| escape sequence allows to test the category code of tokens,
and match control sequences. Each character category is represented
by a single uppercase letter:
\begin{itemize}
\item |C| for control sequences;
\item |B| for begin-group tokens;
\item |E| for end-group tokens;
\item |M| for math shift;
\item |T| for alignment tab tokens;
\item |P| for macro parameter tokens;
\item |U| for superscript tokens (up);
\item |D| for subscript tokens (down);
\item |S| for spaces;
\item |L| for letters;
\item |O| for others; and
\item |A| for active characters.
\end{itemize}
The |\c| escape sequence is used as follows.
\begin{l3regex-syntax}
  \item[\\c\Arg{regex}] A control sequence whose csname matches the
    \meta{regex}, anchored at the beginning and end, so that |\c{begin}|
    matches exactly \cs[no-index]{begin}, and nothing else.
  \item[\\cX] Applies to the next object, which can be a character,
    escape character sequence such as |\x{0A}|, character class, or
    group, and forces this object to only match tokens with category
    |X| (any of |CBEMTPUDSLOA|. For instance, |\cL[A-Z\d]| matches
    uppercase letters and digits of category code letter, |\cC.|
    matches any control sequence, and |\cO(abc)| matches |abc| where
    each character has category other.\footnote{This last example also
    captures \enquote{\texttt{abc}} as a regex group; to avoid this
    use a non-capturing group \texttt{\textbackslash cO(?:abc)}.}
  \item[{\\c[XYZ]}] Applies to the next object, and forces it to only
    match tokens with category |X|, |Y|, or |Z| (each being any of
    |CBEMTPUDSLOA|). For instance, |\c[LSO](..)| matches two tokens of
    category letter, space, or other.
  \item[{\\c[\char`\^XYZ]}] Applies to the next object and prevents it
    from matching any token with category |X|, |Y|, or |Z| (each being
    any of |CBEMTPUDSLOA|). For instance, |\c[^O]\d| matches digits
    which have any category different from other.
\end{l3regex-syntax}
The category code tests can be used inside classes; for instance,
|[\cO\d \c[LO][A-F]]| matches what \TeX{} considers as hexadecimal
digits, namely digits with category other, or uppercase letters from
|A| to |F| with category either letter or other. Within a group
affected by a category code test, the outer test can be overridden by
a nested test: for instance, |\cL(ab\cO\*cd)| matches |ab*cd| where
all characters are of category letter, except |*| which has category
other.

The |\u| escape sequence allows to insert the contents of a token list
directly into a regular expression or a replacement, avoiding the need
to escape special characters. Namely, |\u|\Arg{var name} matches
the exact contents (both character codes and category codes) of the
variable \cs[no-index]{\meta{var name}}. %,
%which are obtained by applying \cs{exp_not:v} \Arg{var name} at the
%time the regular expression is compiled.
Within a |\c{...}|
control sequence matching, the |\u| escape sequence only expands its
argument once. %, in effect performing \cs{tl_to_str:v}.
Quantifiers are supported.

The |\ur| escape sequence allows to insert the contents of a |regex|
variable into a larger regular expression.  For instance,
|A\ur{lTmpaRegex}D| matches the tokens |A| and |D| separated by
something that matches the regular expression
\cs{lTmpaRegex}.  This behaves as if a non-capturing group
were surrounding \cs{lTmpaRegex}, and any group contained
in \cs{lTmpaRegex} is converted to a non-capturing group.
Quantifiers are supported.

For instance, if \cs{lTmpaRegex} has value \verb"B|C",
then |A\ur{l_tmpa_regex}D| is equivalent to \verb"A(?:B|C)D" (matching
|ABD| or |ACD|) and not to \verb"AB|CD" (matching |AB| or |CD|).  To
get the latter effect, it is simplest to use \TeX{}'s expansion
machinery directly: if \cs{lTmpaTl} contains
\verb"B|C" then the following two lines show the same result:
\begin{codehigh}
\regexShow {A \u{lTmpaTl} D}
\regexShow {A B | C D}
\end{codehigh}

\subsection{Miscellaneous}

Anchors and simple assertions.
\begin{l3regex-syntax}
  \item[\\b] Word boundary: either the previous token is matched by
    |\w| and the next by |\W|, or the opposite. For this purpose,
    the ends of the token list are considered as |\W|.
  \item[\\B] Not a word boundary: between two |\w| tokens
    or two |\W| tokens (including the boundary).
  \item[\char`^ \textrm{or} \\A]
    Start of the subject token list.
  \item[\char`$\textrm{,} \\Z \textrm{or} \\z] %^^A $
    End of the subject token list.
  \item[\\G] Start of the current match. This is only different from |^|
    in the case of multiple matches: for instance
    |\regexCount {\G a} {aaba} \lTmpaInt| yields $2$, but
    replacing |\G| by |^| would result in \cs{lTmpaInt} holding the
    value $1$.
\end{l3regex-syntax}

The option |(?i)| makes the match case insensitive (identifying
\texttt{A}--\texttt{Z} with \texttt{a}--\texttt{z}; no Unicode support
yet). This applies until the end of the group in which it appears, and
can be reverted using |(?-i)|. For instance, in
\verb"(?i)(a(?-i)b|c)d", the letters |a| and |d| are affected by the
|i| option. Characters within ranges and classes are affected
individually: |(?i)[Y-\\]| is equivalent to |[YZ\[\\yz]|, and
|(?i)[^aeiou]| matches any character which is not a vowel. Neither
character properties, nor |\c{...}| nor |\u{...}| are affected by the
|i| option.
%^^A \]

\section{Syntax of the Replacement Text}

Most of the features described in regular expressions do not make
sense within the replacement text.  Backslash introduces various
special constructions, described further below:
\begin{itemize}
  \item |\0| is the whole match;
  \item |\1| is the submatch that was matched by the first (capturing)
    group |(...)|; similarly for |\2|, \ldots{}, |\9| and
    |\g{|\meta{number}|}|;
  \item \verb*|\ | inserts a space (spaces are ignored when not
    escaped);
  \item |\a|, |\e|, |\f|, |\n|, |\r|, |\t|, |\xhh|, |\x{hhh}|
    correspond to single characters as in regular expressions;
  \item |\c|\Arg{cs name} inserts a control sequence;
  \item |\c|\meta{category}\meta{character} (see below);
  \item |\u|\Arg{tl var name} inserts the contents of the
    \meta{tl var} (see below).
\end{itemize}
Characters other than backslash and space are simply inserted in the
result (but since the replacement text is first converted to a string,
one should also escape characters that are special for \TeX{}, for
instance use |\#|).  Non-alphanumeric characters can always be safely
escaped with a backslash.

For instance,
\begin{demohigh}
\tlSet \lTmpaTl {Hello, world!}
\regexReplaceAll {([er]?l|o) .} {(\0--\1)} \lTmpaTl
\tlUse \lTmpaTl
\end{demohigh}

The submatches are numbered according to the order in which the
opening parenthesis of capturing groups appear in the regular
expression to match.  The $n$-th submatch is empty if there are fewer
than $n$ capturing groups or for capturing groups that appear in
alternatives that were not used for the match.  In case a capturing
group matches several times during a match (due to quantifiers) only
the last match is used in the replacement text. Submatches always keep
the same category codes as in the original token list.

By default, the category code of characters inserted by the
replacement are determined by the prevailing category code regime at
the time where the replacement is made, with two exceptions:
\begin{itemize}
\item space characters (with character code $32$) inserted with
  \verb*|\ | or |\x20| or |\x{20}| have category code $10$ regardless
  of the prevailing category code regime;
\item if the category code would be $0$ (escape), $5$ (newline),
  $9$ (ignore), $14$ (comment) or $15$ (invalid), it is replaced by
  $12$ (other) instead.
\end{itemize}
The escape sequence |\c| allows to insert characters
with arbitrary category codes, as well as control sequences.
\begin{l3regex-syntax}
\item[\\cX(\ldots{})] Produces the characters \enquote{\ldots{}} with
  category |X|, which must be one of |CBEMTPUDSLOA| as in regular
  expressions.  Parentheses are optional for a single character (which
  can be an escape sequence).  When nested, the innermost category
  code applies, for instance |\cL(Hello\cS\ world)!| gives this text
  with standard category codes.
\item[\\c\Arg{text}] Produces the control sequence with csname
  \meta{text}.  The \meta{text} may contain references to the
  submatches |\0|, |\1|, and so on, as in the example for |\u| below.
\end{l3regex-syntax}

The escape sequence |\u|\Arg{var name} allows to insert the
contents of the variable with name \meta{var name} directly into
the replacement, giving an easier control of category codes.  When
nested in |\c{|\ldots{}|}| and |\u{|\ldots{}|}| constructions, the
|\u| and |\c| escape sequences %perform \cs{tl_to_str:v}, namely
extract the value of the control sequence and turn it into a string.
Matches can also be used within the arguments of |\c| and |\u|.  For
instance,
\begin{demohigh}
\tlSet \lMyOneTl {first}
\tlSet \lMyTwoTl {\underline{second}}
\tlSet \lTmpaTl {One,Two,One,One}
\regexReplaceAll {[^,]+} {\u{lMy\0Tl}} \lTmpaTl
\tlUse \lTmpaTl
\end{demohigh}

Regex replacement is also a convenient way to produce token lists
with arbitrary category codes.  For instance
\begin{codehigh}
\tlClear \lTmpaTl
\regexReplaceAll { } {\cU\% \cA\~} \lTmpaTl
\end{codehigh}
results in \cs{lTmpaTl} containing the percent character
with category code $7$ (superscript) and an active tilde character.

\end{document}
