% -*- coding: utf-8 -*-
% !TEX program = lualatex
\documentclass[oneside]{book}

% -*- coding: utf-8 -*-
% !TEX program = lualatex

\usepackage[a4paper,margin=2.5cm]{geometry}

\newcommand*{\myversion}{2022H}
\newcommand*{\mydate}{Version \myversion\ (\the\year-\mylpad\month-\mylpad\day)}
\newcommand*{\mylpad}[1]{\ifnum#1<10 0\the#1\else\the#1\fi}

\setlength{\parindent}{0pt}
\setlength{\parskip}{4pt plus 1pt minus 1pt}

\usepackage{codehigh}

\colorlet{highback}{blue9}
%\CodeHigh{lite}
\CodeHigh{language=latex/latex2,style/main=highback,style/code=highback}
\NewCodeHighEnv{code}{style/main=gray9,style/code=gray9}
\NewCodeHighEnv{demo}{style/main=gray9,style/code=gray9,demo}

\usepackage{enumitem}

\NewDocumentCommand\MySubScript{m}{$_{#1}$}

\ExplSyntaxOn
\NewDocumentCommand\PrintVarList{m}{
  \clist_set:Nn \l_tmpa_clist {#1}
  \clist_map_inline:Nn \l_tmpa_clist
    {
      \token_to_str:N ##1 ~
    }
}
\NewDocumentCommand\RelaceChacters{m}{
  \tl_set:Nn \lTmpaTl {#1}
  \regex_replace_once:nnN { \_ } { \c{MySubScript} } \lTmpaTl
}
\NewDocumentCommand\RelaceUnderScoreAll{m}{
  \tl_set:Nn \lTmpaTl {#1}
  \regex_replace_all:nnN { \_ } { \c{_} } \lTmpaTl
  \lTmpaTl
}
\ExplSyntaxOff

\NewDocumentEnvironment{variable}{om}{
  \vspace{5pt}
  \begin{minipage}{\linewidth}
  \hrule\vspace{4pt}\obeylines%
  \begingroup
  \ttfamily\bfseries\color{azure3}
  \PrintVarList{#2}
  \endgroup
  \par\vspace{4pt}\hrule
  \end{minipage}\par\nopagebreak\vspace{4pt}
}{%
  \vspace{5pt}%
}

\NewDocumentEnvironment{function}{om}{
  \vspace{5pt}%
}{\vspace{5pt}}

\NewDocumentEnvironment{syntax}{}{%
  \begin{minipage}{\linewidth}
  \hrule\vspace{4pt}\obeylines%
}{%
  \par\vspace{4pt}\hrule
  \end{minipage}\par\nopagebreak\vspace{4pt}
}

\NewDocumentEnvironment{texnote}{}{}{}

\NewDocumentCommand\cs{O{}m}{%
  \texttt{\bfseries\color{purple3}\cBackslashStr\RelaceUnderScoreAll{#2}}%
}
\NewDocumentCommand\meta{m}{%
  \RelaceChacters{#1}%
  \textsl{$\langle$\ignorespaces\lTmpaTl\unskip$\rangle$}%
}
\NewDocumentCommand\Arg{m}{%
  \RelaceChacters{#1}%
  \texttt{\{}\textsl{$\langle$\ignorespaces\lTmpaTl\unskip$\rangle$}\texttt{\}}%
}

\NewDocumentCommand\pkg{m}{\textsf{#1}}

\NewDocumentCommand\nan{}{\texttt{NaN}}
\NewDocumentCommand\enquote{m}{``#1''}

\let\tn=\cs

\RenewDocumentCommand\emph{m}{%
  \underline{\textsl{#1}}%
}

\newenvironment{l3regex-syntax}
  {\begin{itemize}\def\\{\char`\\}\def\makelabel##1{\hss\llap{\ttfamily##1}}}
  {\end{itemize}}

\usepackage{shortvrb}

\NewDocumentCommand \MyMakeShortVerb {} {%
  \MakeShortVerb \|%
  \MakeShortVerb \"%
}
\NewDocumentCommand \MyDeleteShortVerb {} {%
  \DeleteShortVerb \|%
  \DeleteShortVerb \"%
}

\AtBeginDocument{\MyMakeShortVerb}
\AtEndDocument{\MyDeleteShortVerb}
\AddToHook{env/demohigh/before}{\MyDeleteShortVerb}
\AddToHook{env/demohigh/after}{\MyMakeShortVerb}

\usepackage{hologo}
\providecommand*\LuaTeX{\hologo{LuaTeX}}
\providecommand*\pdfTeX{\hologo{pdfTeX}}
\providecommand*\XeTeX{\hologo{XeTeX}}

\usepackage{amsmath}

\usepackage{hyperref}
\hypersetup{
  colorlinks=true,
  urlcolor=blue3,
  linkcolor=blue3,
}

\usepackage{functional}
%\Functional{scoping=false,tracing=true}

\CodeHigh{lite}

\begin{document}

\chapter{Token Manipulation (\texttt{Token})}

\begin{function}{\charLowercase,\charUppercase,\charTitlecase,\charFoldcase}
\begin{syntax}
\cs{charLowercase} \meta{char}
\cs{charUppercase} \meta{char}
\cs{charTitlecase} \meta{char}
\cs{charFoldcase} \meta{char}
\end{syntax}
Converts the \meta{char} to the equivalent case-changed character
as detailed by the function name (see %\cs{strFoldcase} and
\cs{textTitlecase} for details of these terms). The case mapping
is carried out with no context-dependence (\emph{cf.} \cs{textUppercase},
\emph{etc.}) These functions generate characters with the category code
of the \meta{char} (i.e. only the character code changes).
\end{function}

\begin{function}{\charStrLowercase,\charStrUppercase,\charStrTitlecase,\charStrFoldcase}
\begin{syntax}
\cs{charStrLowercase} \meta{char}
\cs{charStrUppercase} \meta{char}
\cs{charStrTitlecase} \meta{char}
\cs{charStrFoldcase} \meta{char}
\end{syntax}
Converts the \meta{char} to the equivalent case-changed character
as detailed by the function name (see %\cs{strFoldcase} and
\cs{textTitlecase} for details of these terms). The case mapping
is carried out with no context-dependence (\emph{cf.} \cs{textUppercase},
\emph{etc.}) These functions generate \enquote{other} (category code $12$)
characters.
\end{function}

\begin{function}{\charSetLccode}
\begin{syntax}
\cs{charSetLccode} \Arg{intexpr_1} \Arg{intexpr_2}
\end{syntax}
Sets up the behaviour of the \meta{character} when
found inside \cs{textLowercase}, such that \meta{character_1}
will be converted into \meta{character_2}. The two \meta{characters}
may be specified using an \meta{integer expression} for the character code
concerned. This may include the \TeX{} \verb|`|\meta{character}
method for converting a single character into its character code:
\begin{codehigh}
\charSetLccode {`\A} {`\a} % Standard behaviour
\charSetLccode {`\A} {`\A + 32}
\charSetLccode {65} {97}
\end{codehigh}
The setting applies within the current \TeX{} group.
\end{function}

\begin{function}{\charSetUccode}
\begin{syntax}
\cs{charSetUccode} \Arg{intexpr_1} \Arg{intexpr_2}
\end{syntax}
Sets up the behaviour of the \meta{character} when
found inside \cs{textUppercase}, such that \meta{character_1}
will be converted into \meta{character_2}. The two \meta{characters}
may be specified using an \meta{integer expression} for the character code
concerned. This may include the \TeX{} \verb|`|\meta{character}
method for converting a single character into its character code:
\begin{codehigh}
\charSetUccode {`\a} {`\A} % Standard behaviour
\charSetUccode {`\a} {`\a - 32}
\charSetUccode {97} {65}
\end{codehigh}
The setting applies within the current \TeX{} group.
\end{function}

\begin{function}{\charValueLccode}
\begin{syntax}
\cs{charValueLccode} \Arg{integer expression}
\end{syntax}
Returns the current lower case code of the \meta{character} with
character code given by the
\meta{integer expression}.
\end{function}

\begin{function}{\charValueUccode}
\begin{syntax}
\cs{charValueUccode} \Arg{integer expression}
\end{syntax}
Returns the current upper case code of the \meta{character} with
character code given by the
\meta{integer expression}.
\end{function}

%\begin{function}{\charShowValueLccode}
%\begin{syntax}
%\cs{charShowValueLccode} \Arg{integer expression}
%\end{syntax}
%Displays the current lower case code of the \meta{character} with
%character code given by the \meta{integer expression} on the
%terminal.
%\end{function}
%
%\begin{function}{\charShowValueUccode}
%\begin{syntax}
%\cs{charShowValueUccode} \Arg{integer expression}
%\end{syntax}
%Displays the current upper case code of the \meta{character} with
%character code given by the \meta{integer expression} on the
%terminal.
%\end{function}

\end{document}
