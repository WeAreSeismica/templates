% -*- coding: utf-8 -*-
% !TEX program = lualatex
\documentclass[oneside]{book}

% -*- coding: utf-8 -*-
% !TEX program = lualatex

\usepackage[a4paper,margin=2.5cm]{geometry}

\newcommand*{\myversion}{2022H}
\newcommand*{\mydate}{Version \myversion\ (\the\year-\mylpad\month-\mylpad\day)}
\newcommand*{\mylpad}[1]{\ifnum#1<10 0\the#1\else\the#1\fi}

\setlength{\parindent}{0pt}
\setlength{\parskip}{4pt plus 1pt minus 1pt}

\usepackage{codehigh}

\colorlet{highback}{blue9}
%\CodeHigh{lite}
\CodeHigh{language=latex/latex2,style/main=highback,style/code=highback}
\NewCodeHighEnv{code}{style/main=gray9,style/code=gray9}
\NewCodeHighEnv{demo}{style/main=gray9,style/code=gray9,demo}

\usepackage{enumitem}

\NewDocumentCommand\MySubScript{m}{$_{#1}$}

\ExplSyntaxOn
\NewDocumentCommand\PrintVarList{m}{
  \clist_set:Nn \l_tmpa_clist {#1}
  \clist_map_inline:Nn \l_tmpa_clist
    {
      \token_to_str:N ##1 ~
    }
}
\NewDocumentCommand\RelaceChacters{m}{
  \tl_set:Nn \lTmpaTl {#1}
  \regex_replace_once:nnN { \_ } { \c{MySubScript} } \lTmpaTl
}
\NewDocumentCommand\RelaceUnderScoreAll{m}{
  \tl_set:Nn \lTmpaTl {#1}
  \regex_replace_all:nnN { \_ } { \c{_} } \lTmpaTl
  \lTmpaTl
}
\ExplSyntaxOff

\NewDocumentEnvironment{variable}{om}{
  \vspace{5pt}
  \begin{minipage}{\linewidth}
  \hrule\vspace{4pt}\obeylines%
  \begingroup
  \ttfamily\bfseries\color{azure3}
  \PrintVarList{#2}
  \endgroup
  \par\vspace{4pt}\hrule
  \end{minipage}\par\nopagebreak\vspace{4pt}
}{%
  \vspace{5pt}%
}

\NewDocumentEnvironment{function}{om}{
  \vspace{5pt}%
}{\vspace{5pt}}

\NewDocumentEnvironment{syntax}{}{%
  \begin{minipage}{\linewidth}
  \hrule\vspace{4pt}\obeylines%
}{%
  \par\vspace{4pt}\hrule
  \end{minipage}\par\nopagebreak\vspace{4pt}
}

\NewDocumentEnvironment{texnote}{}{}{}

\NewDocumentCommand\cs{O{}m}{%
  \texttt{\bfseries\color{purple3}\cBackslashStr\RelaceUnderScoreAll{#2}}%
}
\NewDocumentCommand\meta{m}{%
  \RelaceChacters{#1}%
  \textsl{$\langle$\ignorespaces\lTmpaTl\unskip$\rangle$}%
}
\NewDocumentCommand\Arg{m}{%
  \RelaceChacters{#1}%
  \texttt{\{}\textsl{$\langle$\ignorespaces\lTmpaTl\unskip$\rangle$}\texttt{\}}%
}

\NewDocumentCommand\pkg{m}{\textsf{#1}}

\NewDocumentCommand\nan{}{\texttt{NaN}}
\NewDocumentCommand\enquote{m}{``#1''}

\let\tn=\cs

\RenewDocumentCommand\emph{m}{%
  \underline{\textsl{#1}}%
}

\newenvironment{l3regex-syntax}
  {\begin{itemize}\def\\{\char`\\}\def\makelabel##1{\hss\llap{\ttfamily##1}}}
  {\end{itemize}}

\usepackage{shortvrb}

\NewDocumentCommand \MyMakeShortVerb {} {%
  \MakeShortVerb \|%
  \MakeShortVerb \"%
}
\NewDocumentCommand \MyDeleteShortVerb {} {%
  \DeleteShortVerb \|%
  \DeleteShortVerb \"%
}

\AtBeginDocument{\MyMakeShortVerb}
\AtEndDocument{\MyDeleteShortVerb}
\AddToHook{env/demohigh/before}{\MyDeleteShortVerb}
\AddToHook{env/demohigh/after}{\MyMakeShortVerb}

\usepackage{hologo}
\providecommand*\LuaTeX{\hologo{LuaTeX}}
\providecommand*\pdfTeX{\hologo{pdfTeX}}
\providecommand*\XeTeX{\hologo{XeTeX}}

\usepackage{amsmath}

\usepackage{hyperref}
\hypersetup{
  colorlinks=true,
  urlcolor=blue3,
  linkcolor=blue3,
}

\usepackage{functional}
%\Functional{scoping=false,tracing=true}

\CodeHigh{lite}

\begin{document}

\chapter{Files (\texttt{File})}

This module provides functions for working with external files.
%Some of these functions apply to an entire file, and have prefix \cs{file},
%while others are used to work with files on a line by line basis and have
%prefix \cs{Ior} (reading) or \cs{Iow} (writing).

It is important to remember that when reading external files \TeX{}
attempts to locate them using both the operating system path and entries in the
\TeX{} file database (most \TeX{} systems use such a database). Thus the
\enquote{current path} for \TeX{} is somewhat broader than that for other
programs.

For functions which expect a \meta{file name} argument, this argument
may contain both literal items and expandable content, which should on
full expansion be the desired file name.
%Active characters (as declared in \cs{l_char_active_seq}) are \emph{not}
%expanded, allowing the direct use of these in file names.
Quote tokens (\verb|"|) are not permitted in file names as they are reserved
for internal use by some \TeX{} primitives.

Spaces are trimmed at the beginning and end of the file name:
this reflects the fact that some file systems do not allow or interact
unpredictably with spaces in these positions. When no extension is given,
this will trim spaces from the start of the name only.

\section{File Operation Functions}

\begin{function}{\fileInput}
\begin{syntax}
\cs{fileInput} \Arg{file name}
\end{syntax}
Searches for \meta{file name} in the path as detailed for
\cs{fileIfExistTF}, and if found reads in the file and
returns the contents. All files read are recorded
for information and the file name stack is updated by this
function. An error is raised if the file is not found.
\end{function}

\begin{function}{\fileIfExistInput,\fileIfExistInputF}
\begin{syntax}
\cs{fileIfExistInput} \Arg{file name}
\cs{fileIfExistInputF} \Arg{file name} \Arg{false code}
\end{syntax}
Searches for \meta{file name} using the current \TeX{} search path.
%and the additional paths included in \cs{l_file_search_path_seq}.
If found then reads in the file and returns the contents as described
for \cs{fileInput}, otherwise inserts the \meta{false code}.
Note that these functions do not raise
an error if the file is not found, in contrast to \cs{fileInput}.
\end{function}

\begin{function}{\fileGet,\fileGetT,\fileGetF,\fileGetTF}
\begin{syntax}
\cs{fileGet} \Arg{filename} \Arg{setup} \meta{tl}
\cs{fileGetT} \Arg{filename} \Arg{setup} \meta{tl} \Arg{true code}
\cs{fileGetF} \Arg{filename} \Arg{setup} \meta{tl} \Arg{false code}
\cs{fileGetTF} \Arg{filename} \Arg{setup} \meta{tl} \Arg{true code} \Arg{false code}
\end{syntax}
Defines \meta{tl} to the contents of \meta{filename}.
Category codes may need to be set appropriately via the \meta{setup}
argument.
The non-branching version sets the \meta{tl} to \cs{qNoValue} if the file is
not found. The branching version runs the \meta{true code} after the
assignment to \meta{tl} if the file is found, and \meta{false code}
otherwise.
\end{function}

\begin{function}{\fileIfExist,\fileIfExistT,\fileIfExistF,\fileIfExistTF}
\begin{syntax}
\cs{fileIfExist} \Arg{file name}
\cs{fileIfExistT} \Arg{file name} \Arg{true code}
\cs{fileIfExistF} \Arg{file name} \Arg{false code}
\cs{fileIfExistTF} \Arg{file name} \Arg{true code} \Arg{false code}
\end{syntax}
Searches for \meta{file name} using the current \TeX{} search path.
%and the additional paths controlled by \cs{l_file_search_path_seq}.
\end{function}

%\begin{function}{\fileInputStop}
%\begin{syntax}
%\cs{fileInputStop}
%\end{syntax}
%Ends the reading of a file started by \cs{file_input:n} or similar before
%the end of the file is reached. Where the file reading is being terminated
%due to an error, \cs{msg_critical:nn(nn)}
%should be preferred.
%\begin{texnote}
%This function must be used on a line on its own: \TeX{} reads files
%line-by-line and so any additional tokens in the \enquote{current} line
%will still be read.
%\par
%This is also true if the function is hidden inside another function
%(which will be the normal case), i.e., all tokens on the same line
%in the source file are still processed. Putting it on a line by itself
%in the definition doesn't help as it is the line where it is used that
%counts!
%\end{texnote}
%\end{function}

\end{document}
