% -*- coding: utf-8 -*-
% !TEX program = lualatex
\documentclass[oneside]{book}

% -*- coding: utf-8 -*-
% !TEX program = lualatex

\usepackage[a4paper,margin=2.5cm]{geometry}

\newcommand*{\myversion}{2022H}
\newcommand*{\mydate}{Version \myversion\ (\the\year-\mylpad\month-\mylpad\day)}
\newcommand*{\mylpad}[1]{\ifnum#1<10 0\the#1\else\the#1\fi}

\setlength{\parindent}{0pt}
\setlength{\parskip}{4pt plus 1pt minus 1pt}

\usepackage{codehigh}

\colorlet{highback}{blue9}
%\CodeHigh{lite}
\CodeHigh{language=latex/latex2,style/main=highback,style/code=highback}
\NewCodeHighEnv{code}{style/main=gray9,style/code=gray9}
\NewCodeHighEnv{demo}{style/main=gray9,style/code=gray9,demo}

\usepackage{enumitem}

\NewDocumentCommand\MySubScript{m}{$_{#1}$}

\ExplSyntaxOn
\NewDocumentCommand\PrintVarList{m}{
  \clist_set:Nn \l_tmpa_clist {#1}
  \clist_map_inline:Nn \l_tmpa_clist
    {
      \token_to_str:N ##1 ~
    }
}
\NewDocumentCommand\RelaceChacters{m}{
  \tl_set:Nn \lTmpaTl {#1}
  \regex_replace_once:nnN { \_ } { \c{MySubScript} } \lTmpaTl
}
\NewDocumentCommand\RelaceUnderScoreAll{m}{
  \tl_set:Nn \lTmpaTl {#1}
  \regex_replace_all:nnN { \_ } { \c{_} } \lTmpaTl
  \lTmpaTl
}
\ExplSyntaxOff

\NewDocumentEnvironment{variable}{om}{
  \vspace{5pt}
  \begin{minipage}{\linewidth}
  \hrule\vspace{4pt}\obeylines%
  \begingroup
  \ttfamily\bfseries\color{azure3}
  \PrintVarList{#2}
  \endgroup
  \par\vspace{4pt}\hrule
  \end{minipage}\par\nopagebreak\vspace{4pt}
}{%
  \vspace{5pt}%
}

\NewDocumentEnvironment{function}{om}{
  \vspace{5pt}%
}{\vspace{5pt}}

\NewDocumentEnvironment{syntax}{}{%
  \begin{minipage}{\linewidth}
  \hrule\vspace{4pt}\obeylines%
}{%
  \par\vspace{4pt}\hrule
  \end{minipage}\par\nopagebreak\vspace{4pt}
}

\NewDocumentEnvironment{texnote}{}{}{}

\NewDocumentCommand\cs{O{}m}{%
  \texttt{\bfseries\color{purple3}\cBackslashStr\RelaceUnderScoreAll{#2}}%
}
\NewDocumentCommand\meta{m}{%
  \RelaceChacters{#1}%
  \textsl{$\langle$\ignorespaces\lTmpaTl\unskip$\rangle$}%
}
\NewDocumentCommand\Arg{m}{%
  \RelaceChacters{#1}%
  \texttt{\{}\textsl{$\langle$\ignorespaces\lTmpaTl\unskip$\rangle$}\texttt{\}}%
}

\NewDocumentCommand\pkg{m}{\textsf{#1}}

\NewDocumentCommand\nan{}{\texttt{NaN}}
\NewDocumentCommand\enquote{m}{``#1''}

\let\tn=\cs

\RenewDocumentCommand\emph{m}{%
  \underline{\textsl{#1}}%
}

\newenvironment{l3regex-syntax}
  {\begin{itemize}\def\\{\char`\\}\def\makelabel##1{\hss\llap{\ttfamily##1}}}
  {\end{itemize}}

\usepackage{shortvrb}

\NewDocumentCommand \MyMakeShortVerb {} {%
  \MakeShortVerb \|%
  \MakeShortVerb \"%
}
\NewDocumentCommand \MyDeleteShortVerb {} {%
  \DeleteShortVerb \|%
  \DeleteShortVerb \"%
}

\AtBeginDocument{\MyMakeShortVerb}
\AtEndDocument{\MyDeleteShortVerb}
\AddToHook{env/demohigh/before}{\MyDeleteShortVerb}
\AddToHook{env/demohigh/after}{\MyMakeShortVerb}

\usepackage{hologo}
\providecommand*\LuaTeX{\hologo{LuaTeX}}
\providecommand*\pdfTeX{\hologo{pdfTeX}}
\providecommand*\XeTeX{\hologo{XeTeX}}

\usepackage{amsmath}

\usepackage{hyperref}
\hypersetup{
  colorlinks=true,
  urlcolor=blue3,
  linkcolor=blue3,
}

\usepackage{functional}
%\Functional{scoping=false,tracing=true}

\CodeHigh{lite}

\begin{document}

\chapter{Functional Progarmming (\texttt{Prg})}

\section{Defining Functions and Conditionals}

\begin{function}{\prgNewFunction}
\begin{syntax}
\cs{prgNewFunction} \meta{function} \Arg{argument specification} \Arg{code}
\end{syntax}
Creates protected \meta{function} for evaluating the \meta{code}.
Within the \meta{code}, the parameters (\verb|#1|, \verb|#2|,
\emph{etc.}) will be replaced by those absorbed by the function.
The returned value \emph{must} be passed with \cs{prgReturn} function.
The definition is global and an error results if the
\meta{function} is already defined.\par
The \Arg{argument specification} in a list of letters,
where each letter is one of the following argument specifiers
(nearly all of them are \texttt{M} or \texttt{m} for functions provided by this package):\par
{\centering\begin{tabular}{ll}
%\hline
  \texttt{M} & single-token argument, which will be manipulated first \\
  \texttt{m} & multi-token argument, which will be manipulated first \\
  \texttt{N} & single-token argument, which will not be manipulated first \\
  \texttt{n} & multi-token argument, which will not be manipulated first \\
%\hline
\end{tabular}\par}
The argument manipulation for argument type \texttt{M} or \texttt{m}
is: if the argument starts with a function defined with \cs{prgNewFunction},
the argument will be evaluated and replaced with the returned value.
\end{function}

\begin{function}{\prgSetEqFunction}
\begin{syntax}
\cs{prgSetEqFunction} \meta{function_1} \meta{function_2}
\end{syntax}
Sets \meta{function_1} as an alias of \meta{function_2}.
\end{function}

\begin{function}{\prgNewConditional}
\begin{syntax}
\cs{prgNewConditional} \meta{function} \Arg{argument specification} \Arg{code}
\end{syntax}
Creates protected conditional \meta{function} for evaluating the \meta{code}.
The returned value of the \meta{function} \emph{must} be either |\cTrueBool|
or |\cFalseBool| and be passed with \cs{prgReturn} function..
The definition is global and an error results if the \meta{function} is already defined.
\par
Assume the \meta{function} is |\fooIfBar|, then another three functions
are also created at the same time: |\fooIfBarT|, |\fooIfBarF|, and |\fooIfBarTF|.
They have extra arguments which are \Arg{true code} or/and \Arg{false code}.
For example, if you write
\begin{codehigh}
\prgNewConditional \fooIfBar {Mm} {code with return value \cTrueBool or \cFalseBool}
\end{codehigh}
Then the following four functions are created:
\begin{itemize}
 \item |\fooIfBar| \meta{arg_1} \Arg{arg_2}
 \item |\fooIfBarT| \meta{arg_1} \Arg{arg_2} \Arg{true code}
 \item |\fooIfBarF| \meta{arg_1} \Arg{arg_2} \Arg{false code}
 \item |\fooIfBarTF| \meta{arg_1} \Arg{arg_2} \Arg{true code} \Arg{false code}
\end{itemize}
\end{function}

\section{Returning Values and Printing Tokens}

Just like \LuaTeX, \pkg{functional} package also provides \cs{prgReturn} and \cs{prgPrint} functions.

\begin{function}{\prgReturn}
\begin{syntax}
\cs{prgReturn} \Arg{tokens}
\end{syntax}
Returns \meta{tokens} as result of current function or conditional.
This function is normally used in the \meta{code} of \cs{prgNewFunction}
or \cs{prgNewConditional}, and it \emph{must} be the last function evaluated in the \meta{code}.
If it is missing, the return value of the last function evaluated in the \meta{code}
is returned. Therefore, the following two examples produce the same output:
\begin{codehigh}
\IgnoreSpacesOn
\prgNewFunction \mathSquare { m } {
  \intSet \lTmpaInt {\intEval {#1 * #1}}
  \prgReturn {\expValue \lTmpaInt}
}
\IgnoreSpacesOff
\mathSquare{5}
\end{codehigh}
\begin{codehigh}
\IgnoreSpacesOn
\prgNewFunction \mathSquare { m } {
  \intSet \lTmpaInt {\intEval {#1 * #1}}
  \expValue \lTmpaInt
}
\IgnoreSpacesOff
\mathSquare{5}
\end{codehigh}
\pkg{Functional} package takes care of return values,
and only print them to the input stream if the outer most functions are evaluated.
\end{function}

\begin{function}{\prgPrint}
\begin{syntax}
\cs{prgPrint} \Arg{tokens}
\end{syntax}
Prints \meta{tokens} directly to the input stream.
If there is no function defined with \cs{prgNewFunction} in \meta{tokens},
you can omit \cs{prgPrint} and write only \meta{tokens}.
But if there is any function defined with \cs{prgNewFunction} in \meta{tokens},
you \emph{have to} use \cs{prgPrint} function.
\end{function}

\section{Running Code with Anonymous Functions}

\begin{function}{\prgDo}
\begin{syntax}
\cs{prgDo} \Arg{code}
\end{syntax}
Treats \meta{code} as an anonymous function without arguments and evaluates it.
\end{function}

\begin{function}{\prgRunOneArgCode,\prgRunTwoArgCode,\prgRunThreeArgCode,\prgRunFourArgCode}
\begin{syntax}
\cs{prgRunOneArgCode} \Arg{arg_1} \Arg{code}
\cs{prgRunTwoArgCode} \Arg{arg_1} \Arg{arg_2} \Arg{code}
\cs{prgRunThreeArgCode} \Arg{arg_1} \Arg{arg_2} \Arg{arg_3} \Arg{code}
\cs{prgRunFourArgCode} \Arg{arg_1} \Arg{arg_2} \Arg{arg_3} \Arg{arg_4} \Arg{code}
\end{syntax}
Treats \meta{code} as an anonymous function with one to four arguments respectively,
and evaluates it. In evaluating the \meta{code}, \pkg{functional} package first evaluates
\meta{arg_1} to \meta{arg_4}, then replaces |#1| to |#4| in \meta{code}
with the return values respectively.
\end{function}

\end{document}
