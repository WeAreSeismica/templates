% -*- coding: utf-8 -*-
% !TEX program = lualatex
\documentclass[oneside]{book}

% -*- coding: utf-8 -*-
% !TEX program = lualatex

\usepackage[a4paper,margin=2.5cm]{geometry}

\newcommand*{\myversion}{2022H}
\newcommand*{\mydate}{Version \myversion\ (\the\year-\mylpad\month-\mylpad\day)}
\newcommand*{\mylpad}[1]{\ifnum#1<10 0\the#1\else\the#1\fi}

\setlength{\parindent}{0pt}
\setlength{\parskip}{4pt plus 1pt minus 1pt}

\usepackage{codehigh}

\colorlet{highback}{blue9}
%\CodeHigh{lite}
\CodeHigh{language=latex/latex2,style/main=highback,style/code=highback}
\NewCodeHighEnv{code}{style/main=gray9,style/code=gray9}
\NewCodeHighEnv{demo}{style/main=gray9,style/code=gray9,demo}

\usepackage{enumitem}

\NewDocumentCommand\MySubScript{m}{$_{#1}$}

\ExplSyntaxOn
\NewDocumentCommand\PrintVarList{m}{
  \clist_set:Nn \l_tmpa_clist {#1}
  \clist_map_inline:Nn \l_tmpa_clist
    {
      \token_to_str:N ##1 ~
    }
}
\NewDocumentCommand\RelaceChacters{m}{
  \tl_set:Nn \lTmpaTl {#1}
  \regex_replace_once:nnN { \_ } { \c{MySubScript} } \lTmpaTl
}
\NewDocumentCommand\RelaceUnderScoreAll{m}{
  \tl_set:Nn \lTmpaTl {#1}
  \regex_replace_all:nnN { \_ } { \c{_} } \lTmpaTl
  \lTmpaTl
}
\ExplSyntaxOff

\NewDocumentEnvironment{variable}{om}{
  \vspace{5pt}
  \begin{minipage}{\linewidth}
  \hrule\vspace{4pt}\obeylines%
  \begingroup
  \ttfamily\bfseries\color{azure3}
  \PrintVarList{#2}
  \endgroup
  \par\vspace{4pt}\hrule
  \end{minipage}\par\nopagebreak\vspace{4pt}
}{%
  \vspace{5pt}%
}

\NewDocumentEnvironment{function}{om}{
  \vspace{5pt}%
}{\vspace{5pt}}

\NewDocumentEnvironment{syntax}{}{%
  \begin{minipage}{\linewidth}
  \hrule\vspace{4pt}\obeylines%
}{%
  \par\vspace{4pt}\hrule
  \end{minipage}\par\nopagebreak\vspace{4pt}
}

\NewDocumentEnvironment{texnote}{}{}{}

\NewDocumentCommand\cs{O{}m}{%
  \texttt{\bfseries\color{purple3}\cBackslashStr\RelaceUnderScoreAll{#2}}%
}
\NewDocumentCommand\meta{m}{%
  \RelaceChacters{#1}%
  \textsl{$\langle$\ignorespaces\lTmpaTl\unskip$\rangle$}%
}
\NewDocumentCommand\Arg{m}{%
  \RelaceChacters{#1}%
  \texttt{\{}\textsl{$\langle$\ignorespaces\lTmpaTl\unskip$\rangle$}\texttt{\}}%
}

\NewDocumentCommand\pkg{m}{\textsf{#1}}

\NewDocumentCommand\nan{}{\texttt{NaN}}
\NewDocumentCommand\enquote{m}{``#1''}

\let\tn=\cs

\RenewDocumentCommand\emph{m}{%
  \underline{\textsl{#1}}%
}

\newenvironment{l3regex-syntax}
  {\begin{itemize}\def\\{\char`\\}\def\makelabel##1{\hss\llap{\ttfamily##1}}}
  {\end{itemize}}

\usepackage{shortvrb}

\NewDocumentCommand \MyMakeShortVerb {} {%
  \MakeShortVerb \|%
  \MakeShortVerb \"%
}
\NewDocumentCommand \MyDeleteShortVerb {} {%
  \DeleteShortVerb \|%
  \DeleteShortVerb \"%
}

\AtBeginDocument{\MyMakeShortVerb}
\AtEndDocument{\MyDeleteShortVerb}
\AddToHook{env/demohigh/before}{\MyDeleteShortVerb}
\AddToHook{env/demohigh/after}{\MyMakeShortVerb}

\usepackage{hologo}
\providecommand*\LuaTeX{\hologo{LuaTeX}}
\providecommand*\pdfTeX{\hologo{pdfTeX}}
\providecommand*\XeTeX{\hologo{XeTeX}}

\usepackage{amsmath}

\usepackage{hyperref}
\hypersetup{
  colorlinks=true,
  urlcolor=blue3,
  linkcolor=blue3,
}

\usepackage{functional}
%\Functional{scoping=false,tracing=true}

\CodeHigh{lite}

\begin{document}

\chapter{Comma Separated Lists (\texttt{Clist})}

\section{Constant and Scratch Comma Lists}

\begin{variable}{\cEmptyClist}
Constant that is always empty.
\end{variable}

\begin{variable}{\lTmpaClist,\lTmpbClist,\lTmpcClist,\lTmpiClist,\lTmpjClist,\lTmpkClist}
Scratch comma lists for local assignment. These are never used by
the \verb!functional! package, and so are safe for use with any
function. However, they may be overwritten by other
code and so should only be used for short-term storage.
\end{variable}

\begin{variable}{\gTmpaClist,\gTmpbClist,\gTmpcClist,\gTmpiClist,\gTmpjClist,\gTmpkClist}
Scratch comma lists for global assignment. These are never used by
the \verb!functional! package, and so are safe for use with any
function. However, they may be overwritten by other
code and so should only be used for short-term storage.
\end{variable}

\section{Creating and Using Comma Lists}

\begin{function}{\clistNew}
\begin{syntax}
\cs{clistNew} \meta{comma list}
\end{syntax}
Creates a new \meta{comma list} or raises an error if the name is
already taken. The declaration is global. The \meta{comma list}
initially contains no items.
\begin{codehigh}
\clistNew \lFooSomeClist
\end{codehigh}
\end{function}

\begin{function}{\clistConst}
\begin{syntax}
\cs{clistConst} \meta{clist var} \Arg{comma list}
\end{syntax}
Creates a new constant \meta{clist var} or raises an error
if the name is already taken. The value of the
\meta{clist var} is set globally to the
\meta{comma list}.
\begin{codehigh}
\clistConst \cFooSomeClist {one,two,three}
\end{codehigh}
\end{function}

\begin{function}{\clistVarJoin}
\begin{syntax}
\cs{clistVarJoin} \meta{clist var} \Arg{separator}
\end{syntax}
Returns the contents of the \meta{clist var},
with the \meta{separator} between the items.
%If the comma list has a single item, it is placed in the input stream,
%and a comma list with no items produces no output.
%An error is raised if the variable does not exist or if it is invalid.
\begin{demohigh}
\clistSet \lTmpaClist { a , b , , c , {de} , f }
\clistVarJoin \lTmpaClist { and }
\end{demohigh}
%\begin{texnote}
%The result is returned within the \tn{unexpanded}
%primitive (\cs{exp_not:n}), which means that the \meta{items}
%do not expand further when appearing in an \texttt{x}-type
%or \texttt{e}-type argument expansion.
%\end{texnote}
\end{function}

\begin{function}{\clistVarJoinExtended}
\begin{syntax}
\cs{clistVarJoinExtended} \meta{clist var} \Arg{separator between two} \Arg{separator between more than two} \Arg{separator between final two}
\end{syntax}
Returns the contents of the \meta{clist var},
with the appropriate \meta{separator} between the items. Namely, if
the comma list has more than two items, the \meta{separator between
more than two} is placed between each pair of items except the
last, for which the \meta{separator between final two} is used.  If
the comma list has exactly two items, then they are joined with
the \meta{separator between two} and returns.
%If the comma list has a single item, it is placed in the input stream,
%and a comma list with no items produces no output.
%An error is raised if the variable does not exist or if it is invalid.
\begin{demohigh}
\clistSet \lTmpaClist { a , b }
\clistVarJoinExtended \lTmpaClist { and } {, } {, and }
\end{demohigh}
\begin{demohigh}
\clistSet \lTmpaClist { a , b , , c , {de} , f }
\clistVarJoinExtended \lTmpaClist { and } {, } {, and }
\end{demohigh}
%\begin{texnote}
%The result is returned within the \tn{unexpanded}
%primitive (\cs{exp_not:n}), which means that the \meta{items}
%do not expand further when appearing in an \texttt{x}-type
%or \texttt{e}-type argument expansion.
%\end{texnote}
\end{function}

\begin{function}{\clistJoin,\clistJoinExtended}
\begin{syntax}
\cs{clistJoin} \meta{comma list} \Arg{separator}
\cs{clistJoinExtended} \meta{comma list} \Arg{separator between two} \Arg{separator between more than two} \Arg{separator between final two}
\end{syntax}
Returns the contents of the \meta{comma list},
with the appropriate \meta{separator} between the items. As for
\cs{clistSet}, blank items are omitted, spaces are removed from
both sides of each item, then a set of braces is removed if the
resulting space-trimmed item is braced.  The \meta{separators} are
then inserted in the same way as for \cs{clistVarJoin} and
\cs{clistVarJoinExtended}, respectively.
\begin{demohigh}
\clistJoinExtended { a , b } { and } {, } {, and }
\end{demohigh}
\begin{demohigh}
\clistJoinExtended { a , b , , c , {de} , f } { and } {, } {, and }
\end{demohigh}
%\begin{texnote}
%The result is returned within the \tn{unexpanded}
%primitive (\cs{exp_not:n}), which means that the \meta{items}
%do not expand further when appearing in an \texttt{x}-type
%or \texttt{e}-type argument expansion.
%\end{texnote}
\end{function}

\section{Viewing Comma Lists}

\begin{function}{\clistLog}
\begin{syntax}
\cs{clistLog} \Arg{tokens}
\end{syntax}
Writes the entries in the comma list in the log file. See also
\cs{clistShow} which displays the result in the terminal.
\begin{codehigh}
\clistLog {one,two,three}
\end{codehigh}
\end{function}

\begin{function}{\clistVarLog}
\begin{syntax}
\cs{clistVarLog} \meta{comma list}
\end{syntax}
Writes the entries in the \meta{comma list} in the log file. See
also \cs{clistVarShow} which displays the result in the terminal.
\begin{codehigh}
\clistSet \lTmpaClist {one,two,three}
\clistVarLog \lTmpaClist
\end{codehigh}
\end{function}

\begin{function}{\clistShow}
\begin{syntax}
\cs{clistShow} \Arg{tokens}
\end{syntax}
Displays the entries in the comma list in the terminal.
\begin{codehigh}
\clistShow {one,two,three}
\end{codehigh}
\end{function}

\begin{function}{\clistVarShow}
\begin{syntax}
\cs{clistVarShow} \meta{comma list}
\end{syntax}
Displays the entries in the \meta{comma list} in the terminal.
\begin{codehigh}
\clistSet \lTmpaClist {one,two,three}
\clistVarShow \lTmpaClist
\end{codehigh}
\end{function}

\section{Setting Comma Lists}

\begin{function}{\clistSet}
\begin{syntax}
\cs{clistSet} \meta{comma list} \verb|{|\meta{item_1},\ldots{},\meta{item_n}\verb|}|
\end{syntax}
Sets \meta{comma list} to contain the \meta{items},
removing any previous content from the variable.
Blank items are omitted, spaces are removed from both sides of each
item, then a set of braces is removed if the resulting space-trimmed
item is braced.
To store some \meta{tokens} as a single \meta{item} even if the
\meta{tokens} contain commas or spaces, add a set of braces:
\cs{clistSet} \meta{comma list} \verb|{| \Arg{tokens} \verb|}|.
\begin{demohigh}
\clistSet \lTmpaClist {one,two,three}
\clistVarJoin \lTmpaClist { and }
\end{demohigh}
\end{function}

\begin{function}{\clistSetEq}
\begin{syntax}
\cs{clistSetEq} \meta{comma list_1} \meta{comma list_2}
\end{syntax}
Sets the content of \meta{comma list_1} equal to that of
\meta{comma list_2}.  To set a token list variable equal to a comma
list variable, use \cs{tlSetEq}.  Conversely, setting a comma
list variable to a token list is unadvisable unless one checks
space-trimming and related issues.
\begin{demohigh}
\clistSet \lTmpaClist {one,two,three,four}
\clistSetEq \lTmpbClist \lTmpaClist
\clistVarJoin \lTmpbClist { and }
\end{demohigh}
\end{function}

\begin{function}{\clistSetFromSeq}
\begin{syntax}
\cs{clistSetFromSeq} \meta{comma list} \meta{sequence}
\end{syntax}
Converts the data in the \meta{sequence} into a \meta{comma list}:
the original \meta{sequence} is unchanged.
Items which contain either spaces or commas are surrounded by braces.
\begin{demohigh}
\seqPutRight \lTmpaSeq {one}
\seqPutRight \lTmpaSeq {two}
\clistSetFromSeq \lTmpaClist \lTmpaSeq
\clistVarJoin \lTmpaClist { and }
\end{demohigh}
\end{function}

\begin{function}{\clistClear}
\begin{syntax}
\cs{clistClear} \meta{comma list}
\end{syntax}
Clears all items from the \meta{comma list}.
\begin{codehigh}
\clistSet \lTmpaClist {one,two,three,four}
\clistClear \lTmpaClist
\end{codehigh}
\end{function}

\begin{function}{\clistClearNew}
\begin{syntax}
\cs{clistClearNew} \meta{comma list}
\end{syntax}
Ensures that the \meta{comma list} exists globally by applying
\cs{clistNew} if necessary, then applies \cs{clistClear} to leave
the list empty.
\begin{demohigh}
\clistClearNew \lFooSomeClist
\clistSet \lFooSomeClist {one,two,three}
\clistVarJoin \lFooSomeClist { and }
\end{demohigh}
\end{function}

\begin{function}{\clistConcat}
\begin{syntax}
\cs{clistConcat} \meta{comma list_1} \meta{comma list_2} \meta{comma list_3}
\end{syntax}
Concatenates the content of \meta{comma list_2} and \meta{comma list_3}
together and saves the result in \meta{comma list_1}. The items in
\meta{comma list_2} are placed at the left side of the new comma list.
\begin{demohigh}
\clistSet \lTmpbClist {one,two}
\clistSet \lTmpcClist {three,four}
\clistConcat \lTmpaClist \lTmpbClist \lTmpcClist
\clistVarJoin \lTmpaClist { + }
\end{demohigh}
\end{function}

\begin{function}{\clistPutLeft}
\begin{syntax}
\cs{clistPutLeft} \meta{comma list} \verb|{|\meta{item_1},\ldots{},\meta{item_n}\verb|}|
\end{syntax}
Appends the \meta{items} to the left of the \meta{comma list}.
Blank items are omitted, spaces are removed from both sides of each
item, then a set of braces is removed if the resulting space-trimmed
item is braced.
To append some \meta{tokens} as a single \meta{item} even if the
\meta{tokens} contain commas or spaces, add a set of braces:
\cs{clistPutLeft} \meta{comma list} \verb|{| \Arg{tokens} \verb|}|.
\begin{demohigh}
\clistSet \lTmpaClist {one,two}
\clistPutLeft \lTmpaClist {zero}
\clistVarJoin \lTmpaClist { and }
\end{demohigh}
\end{function}

\begin{function}{\clistPutRight}
\begin{syntax}
\cs{clistPutRight} \meta{comma list} \verb|{|\meta{item_1},\ldots{},\meta{item_n}\verb|}|
\end{syntax}
Appends the \meta{items} to the right of the \meta{comma list}.
Blank items are omitted, spaces are removed from both sides of each
item, then a set of braces is removed if the resulting space-trimmed
item is braced.
To append some \meta{tokens} as a single \meta{item} even if the
\meta{tokens} contain commas or spaces, add a set of braces:
\cs{clistPutRight} \meta{comma list} \verb|{| \Arg{tokens} \verb|}|.
\begin{demohigh}
\clistSet \lTmpaClist {one,two}
\clistPutRight \lTmpaClist {three}
\clistVarJoin \lTmpaClist { and }
\end{demohigh}
\end{function}

\section{Modifying Comma Lists}

While comma lists are normally used as ordered lists, it may be
necessary to modify the content. The functions here may be used
to update comma lists, while retaining the order of the unaffected
entries.

\begin{function}{\clistVarRemoveDuplicates}
\begin{syntax}
\cs{clistVarRemoveDuplicates} \meta{comma list}
\end{syntax}
Removes duplicate items from the \meta{comma list}, leaving the
left most copy of each item in the \meta{comma list}.  The \meta{item}
comparison takes place on a token basis, as for \cs{tlIfEqTF}.
\begin{demohigh}
\clistSet \lTmpaClist {one,two,one,two,three}
\clistVarRemoveDuplicates \lTmpaClist
\clistVarJoin \lTmpaClist {,}
\end{demohigh}
%\begin{texnote}
%This function iterates through every item in the \meta{comma list} and
%does a comparison with the \meta{items} already checked. It is therefore
%relatively slow with large comma lists.
%Furthermore, it may fail if any of the items in the
%\meta{comma list} contains \verb|{|, \verb|}|, or \verb|#|
%(assuming the usual \TeX{} category codes apply).
%\end{texnote}
\end{function}

\begin{function}{\clistVarRemoveAll}
\begin{syntax}
\cs{clistVarRemoveAll} \meta{comma list} \Arg{item}
\end{syntax}
Removes every occurrence of \meta{item} from the \meta{comma list}.
The \meta{item} comparison takes place on a token basis, as for
\cs{tlIfEqTF}.
\begin{demohigh}
\clistSet \lTmpaClist {one,two,one,two,three}
\clistVarRemoveAll \lTmpaClist {two}
\clistVarJoin \lTmpaClist {,}
\end{demohigh}
%\begin{texnote}
%The function may fail if the \meta{item} contains \verb|{|, \verb|}|, or \verb|#|
%(assuming the usual \TeX{} category codes apply).
%\end{texnote}
\end{function}

\begin{function}{\clistVarReverse}
\begin{syntax}
\cs{clistVarReverse} \meta{comma list}
\end{syntax}
Reverses the order of items stored in the \meta{comma list}.
\begin{demohigh}
\clistSet \lTmpaClist {one,two,one,two,three}
\clistVarReverse \lTmpaClist
\clistVarJoin \lTmpaClist {,}
\end{demohigh}
\end{function}

%\begin{function}{\clistVarSort}
%\begin{syntax}
%\cs{clistVarSort} \meta{clist var} \Arg{comparison code}
%\end{syntax}
%Sorts the items in the \meta{clist var} according to the
%\meta{comparison code}, and assigns the result to
%\meta{clist var}. The details of sorting comparison are
%described in Section \ref{sec:l3sort:mech}.
%\end{function}

\section{Working with the Contents of Comma Lists}

\begin{function}{\clistCount,\clistVarCount}
\begin{syntax}
\cs{clistCount} \Arg{comma list}
\cs{clistVarCount} \meta{comma list}
\end{syntax}
Returns the number of items in the \meta{comma list}
as an \meta{integer denotation}. The total number of items
in a \meta{comma list} includes those which are duplicates,
\emph{i.e.} every item in a \meta{comma list} is counted.
\begin{demohigh}
\clistSet \lTmpaClist {one,two,three,four}
\clistVarCount \lTmpaClist
\end{demohigh}
\end{function}

\begin{function}{\clistItem}
\begin{syntax}
\cs{clistItem} \Arg{comma list} \Arg{integer expression}
\end{syntax}
Indexing items in the \meta{comma list} from $1$ at the top (left), this
function evaluates the \meta{integer expression} and returns the
appropriate item from the comma list. If the
\meta{integer expression} is negative, indexing occurs from the
bottom (right) of the comma list. When the \meta{integer expression}
is larger than the number of items in the \meta{comma list} (as
calculated by \cs{clistCount}) then the function returns nothing.
\begin{demohigh}
\tlSet \lTmpaTl {\clistItem {one,two,three,four} {3}}
\tlUse \lTmpaTl
\end{demohigh}
%\begin{texnote}
%The result is returned within the \tn{unexpanded}
%primitive (\cs{exp_not:n}), which means that the \meta{item}
%does not expand further when appearing in an \texttt{x}-type
%or \texttt{e}-type argument expansion.
%\end{texnote}
\end{function}

\begin{function}{\clistVarItem}
\begin{syntax}
\cs{clistVarItem} \meta{comma list} \Arg{integer expression}
\end{syntax}
Indexing items in the \meta{comma list} from $1$ at the top (left), this
function evaluates the \meta{integer expression} and returns the
appropriate item from the comma list. If the
\meta{integer expression} is negative, indexing occurs from the
bottom (right) of the comma list. When the \meta{integer expression}
is larger than the number of items in the \meta{comma list} (as
calculated by \cs{clistVarCount}) then the function returns nothing.
\begin{demohigh}
\clistSet \lTmpaClist {one,two,three,four}
\tlSet \lTmpaTl {\clistVarItem \lTmpaClist {3}}
\tlUse \lTmpaTl
\end{demohigh}
%\begin{texnote}
%The result is returned within the \tn{unexpanded}
%primitive (\cs{exp_not:n}), which means that the \meta{item}
%does not expand further when appearing in an \texttt{x}-type
%or \texttt{e}-type argument expansion.
%\end{texnote}
\end{function}

\begin{function}{\clistRandItem,\clistVarRandItem}
\begin{syntax}
\cs{clistRandItem} \Arg{comma list}
\cs{clistVarRandItem} \meta{clist var}
\end{syntax}
Selects a pseudo-random item of the \meta{comma list}.
If the \meta{comma list} has no item, the result is empty.
\begin{demohigh}
\tlSet \lTmpaTl {\clistRandItem {one,two,three,four,five,six}}
\tlUse \lTmpaTl
\tlSet \lTmpaTl {\clistRandItem {one,two,three,four,five,six}}
\tlUse \lTmpaTl
\end{demohigh}
%\begin{texnote}
%The result is returned within the \tn{unexpanded}
%primitive (\cs{exp_not:n}), which means that the \meta{item}
%does not expand further when appearing in an \texttt{x}-type
%or \texttt{e}-type argument expansion.
%\end{texnote}
\end{function}

\section{Comma Lists as Stacks}

Comma lists can be used as stacks, where data is pushed to and popped
from the top of the comma list. (The left of a comma list is the top, for
performance reasons.) The stack functions for comma lists are not
intended to be mixed with the general ordered data functions detailed
in the previous section: a comma list should either be used as an
ordered data type or as a stack, but not in both ways.

\begin{function}{\clistGet}
\begin{syntax}
\cs{clistGet} \meta{comma list} \meta{token list variable}
\end{syntax}
Stores the left-most item from the \meta{comma list} in the
\meta{token list variable} without removing it from the
\meta{comma list}. The \meta{token list variable} is assigned locally.
If the \meta{comma list} is empty the
\meta{token list variable} is set to the marker value \cs{qNoValue}.
\begin{demohigh}
\clistSet \lTmpaClist {two,three,four}
\clistGet \lTmpaClist \lTmpaTl
\tlUse \lTmpaTl
\end{demohigh}
\end{function}

\begin{function}{\clistGetT,\clistGetF,\clistGetTF}
\begin{syntax}
\cs{clistGetT} \meta{comma list} \meta{token list variable} \meta{true code}
\cs{clistGetF} \meta{comma list} \meta{token list variable} \meta{false code}
\cs{clistGetTF} \meta{comma list} \meta{token list variable} \meta{true code} \meta{false code}
\end{syntax}
If the \meta{comma list} is empty, leaves the \meta{false code} in the
input stream.  The value of the \meta{token list variable} is
not defined in this case and should not be relied upon.  If the
\meta{comma list} is non-empty, stores the left-most item from the
\meta{comma list} in the \meta{token list variable} without removing
it from the \meta{comma list}.
The \meta{token list variable} is assigned locally.
\begin{demohigh}
\clistSet \lTmpaClist {two,three,four}
\clistGetTF \lTmpaClist \lTmpaTl {\prgReturn{Yes}} {\prgReturn{No}}
\end{demohigh}
\end{function}

\begin{function}{\clistPop}
\begin{syntax}
\cs{clistPop} \meta{comma list} \meta{token list variable}
\end{syntax}
Pops the left-most item from a \meta{comma list} into the
\meta{token list variable}, \emph{i.e.} removes the item from the
comma list and stores it in the \meta{token list variable}.
The assignment of the \meta{token list variable} is local.
If the \meta{comma list} is empty the
\meta{token list variable} is set to the marker value \cs{qNoValue}.
\begin{demohigh}
\clistSet \lTmpaClist {two,three,four}
\clistPop \lTmpaClist \lTmpaTl
\clistVarJoin \lTmpaClist {,}
\end{demohigh}
\end{function}

\begin{function}{\clistPopT,\clistPopF,\clistPopTF}
\begin{syntax}
\cs{clistPopT} \meta{comma list} \meta{token list variable} \Arg{true code}
\cs{clistPopF} \meta{comma list} \meta{token list variable} \Arg{false code}
\cs{clistPopTF} \meta{comma list} \meta{token list variable} \Arg{true code} \Arg{false code}
\end{syntax}
If the \meta{comma list} is empty, leaves the \meta{false code} in the
input stream.  The value of the \meta{token list variable} is
not defined in this case and should not be relied upon.  If the
\meta{comma list} is non-empty, pops the top item from the
\meta{comma list} in the \meta{token list variable}, \emph{i.e.} removes
the item from the \meta{comma list}.
The \meta{token list variable} is assigned locally.
\begin{demohigh}
\clistSet \lTmpaClist {two,three,four}
\clistPopTF \lTmpaClist \lTmpaTl {\prgReturn{Yes}} {\prgReturn{No}}
\end{demohigh}
\end{function}

\begin{function}{\clistPush}
\begin{syntax}
\cs{clistPush} \meta{comma list} \Arg{items}
\end{syntax}
Adds the \Arg{items} to the top of the \meta{comma list}.
Spaces are removed from both sides of each item as for any
\texttt{n}-type comma list.
\begin{demohigh}
\clistSet \lTmpaClist {two,three,four}
\clistPush \lTmpaClist {zero,one}
\clistVarJoin \lTmpaClist {|}
\end{demohigh}
\end{function}

\section{Mapping over Comma Lists}

%The functions described in this section apply a specified function
%to each item of a comma list.
%All mappings are done at the current group level, \emph{i.e.} any
%local assignments made by the \meta{function} or \meta{code} discussed
%below remain in effect after the loop.

When the comma list is given explicitly, %as an \texttt{n}-type argument,
spaces are trimmed around each item.
If the result of trimming spaces is empty, the item is ignored.
Otherwise, if the item is surrounded by braces, one set is removed,
and the result is passed to the mapped function. Thus, if the
comma list that is being mapped is \verb*|{a , {{b} }, ,{}, {c},}|
then the arguments passed to the mapped function are
`\verb*|a|', `\verb*|{b} |', an empty argument, and `\verb*|c|'.

When the comma list is given as a variable, spaces
have already been trimmed on input, and items are simply stripped
of one set of braces if any. This case is more efficient than using
explicit comma lists.

%\begin{function}{\clistMapFunction,\clistVarMapFunction}
%\begin{syntax}
%\cs{clistMapFunction} \Arg{comma list} \meta{function}
%\cs{clistVarMapFunction} \meta{comma list} \meta{function}
%\end{syntax}
%Applies \meta{function} to every \meta{item} stored in the
%\meta{comma list}. The \meta{function} receives one argument for
%each iteration. The \meta{items} are returned from left to right.
%The function \cs{clistMapInline} is in general more efficient
%than \cs{clistMapFunction}.
%\end{function}

\begin{function}{\clistMapInline,\clistVarMapInline}
\begin{syntax}
\cs{clistMapInline} \Arg{comma list} \Arg{inline function}
\cs{clistVarMapInline} \meta{comma list} \Arg{inline function}
\end{syntax}
Applies \meta{inline function} to every \meta{item} stored
within the \meta{comma list}. The \meta{inline function} should
consist of code which receives the \meta{item} as \verb|#1|.
The \meta{items} are returned from left to right.
\begin{demohigh}
\IgnoreSpacesOn
\tlClear \lTmpaTl
\clistMapInline {one,two,three} {
  \tlPutRight \lTmpaTl {(#1)}
}
\tlUse \lTmpaTl
\IgnoreSpacesOff
\end{demohigh}
\end{function}

\begin{function}{\clistMapVariable,\clistVarMapVariable}
\begin{syntax}
\cs{clistMapVariable} \Arg{comma list} \meta{variable} \Arg{code}
\cs{clistVarMapVariable} \meta{comma list} \meta{variable} \Arg{code}
\end{syntax}
Stores each \meta{item} of the \meta{comma list} in turn in the
(token list) \meta{variable} and applies the \meta{code}.  The
\meta{code} will usually make use of the \meta{variable}, but this
is not enforced.  The assignments to the \meta{variable} are local.
Its value after the loop is the last \meta{item} in the \meta{comma
list}, or its original value if there were no \meta{item}.  The
\meta{items} are returned from left to right.
\begin{demohigh}
\IgnoreSpacesOn
\clistMapVariable {one,two,three} \lTmpiTl {
  \tlPutRight \gTmpaTl {\expWhole {(\lTmpiTl)}}
}
\tlUse \gTmpaTl
\IgnoreSpacesOff
\end{demohigh}
\end{function}

%\begin{function}{\clistMapTokens,\clistVarMapTokens}
%\begin{syntax}
%\cs{clistMapTokens} \Arg{comma list} \Arg{code}
%\cs{clistVarMapTokens} \meta{clist var} \Arg{code}
%\end{syntax}
%Calls \meta{code} \Arg{item} for every \meta{item} stored in the
%\meta{comma list}. The \meta{code} receives each \meta{item} as a
%trailing brace group.  If the \meta{code} consists of a single
%function this is equivalent to \cs{clistMapFunction} or \cs{clistVarMapFunction}.
%\end{function}

%\begin{function}{\clistMapBreak}
%\begin{syntax}
%\cs{clistMapBreak}
%\end{syntax}
%Used to terminate a clist map function before all
%entries in the \meta{comma list} have been processed. This
%normally takes place within a conditional statement, for example
%\begin{verbatim}
%\clist_map_inline:Nn \l_my_clist
%{
%\str_if_eq:nnTF { #1 } { bingo }
%{ \clist_map_break: }
%{
%Do something useful
%}
%}
%\end{verbatim}
%Use outside of a clist map scenario leads to low
%level \TeX{} errors.
%\begin{texnote}
%When the mapping is broken, additional tokens may be inserted
%before further items are taken
%from the input stream. This depends on the design of the mapping
%function.
%\end{texnote}
%\end{function}
%
%\begin{function}{\clistMapBreakDo}
%\begin{syntax}
%\cs{clistMapBreakDo} \Arg{code}
%\end{syntax}
%Used to terminate a clist map  function before all
%entries in the \meta{comma list} have been processed, inserting
%the \meta{code} after the mapping has ended. This
%normally takes place within a conditional statement, for example
%\begin{verbatim}
%\clist_map_inline:Nn \l_my_clist
%{
%\str_if_eq:nnTF { #1 } { bingo }
%{ \clist_map_break:n { <code> } }
%{
%Do something useful
%}
%}
%\end{verbatim}
%Use outside of a clist map scenario leads to low
%level \TeX{} errors.
%\begin{texnote}
%When the mapping is broken, additional tokens may be inserted
%before the \meta{code} is
%inserted into the input stream.
%This depends on the design of the mapping function.
%\end{texnote}
%\end{function}

\section{Comma List Conditionals}

\begin{function}{\clistIfExist,\clistIfExistT,\clistIfExistF,\clistIfExistTF}
\begin{syntax}
\cs{clistIfExist} \meta{comma list}
\cs{clistIfExistT} \meta{comma list} \Arg{true code}
\cs{clistIfExistF} \meta{comma list} \Arg{false code}
\cs{clistIfExistTF} \meta{comma list} \Arg{true code} \Arg{false code}
\end{syntax}
Tests whether the \meta{comma list} is currently defined.  This does
not check that the \meta{comma list} really is a comma list.
\begin{demohigh}
\clistIfExistTF \lTmpaClist {\prgReturn{Yes}} {\prgReturn{No}}
\clistIfExistTF \lFooUndefinedClist {\prgReturn{Yes}} {\prgReturn{No}}
\end{demohigh}
\end{function}

\begin{function}{\clistIfEmpty,\clistIfEmptyT,\clistIfEmptyF,\clistIfEmptyTF}
\begin{syntax}
\cs{clistIfEmpty} \Arg{comma list}
\cs{clistIfEmptyT} \Arg{comma list} \Arg{true code}
\cs{clistIfEmptyF} \Arg{comma list} \Arg{false code}
\cs{clistIfEmptyTF} \Arg{comma list} \Arg{true code} \Arg{false code}
\end{syntax}
Tests if the \meta{comma list} is empty (containing no items).
The rules for space trimming are as for other \texttt{n}-type
comma-list functions, hence the comma list \verb|{ , ,, }| (without
outer braces) is empty, while \verb|{ ,{},}| (without outer braces)
contains one element, which happens to be empty: the comma-list
is not empty.
\begin{demohigh}
\clistIfEmptyTF {one,two} {\prgReturn{Empty}} {\prgReturn{NonEmpty}}
\clistIfEmptyTF { , } {\prgReturn{Empty}} {\prgReturn{NonEmpty}}
\end{demohigh}
\end{function}

\begin{function}{\clistVarIfEmpty,\clistVarIfEmptyT,\clistVarIfEmptyF,\clistVarIfEmptyTF}
\begin{syntax}
\cs{clistVarIfEmpty} \meta{comma list}
\cs{clistVarIfEmptyT} \meta{comma list} \Arg{true code}
\cs{clistVarIfEmptyF} \meta{comma list} \Arg{false code}
\cs{clistVarIfEmptyTF} \meta{comma list} \Arg{true code} \Arg{false code}
\end{syntax}
Tests if the \meta{comma list} is empty (containing no items).
\begin{demohigh}
\clistSet \lTmpaClist {one,two}
\clistVarIfEmptyTF \lTmpaClist {\prgReturn{Empty}} {\prgReturn{NonEmpty}}
\clistClear \lTmpaClist
\clistVarIfEmptyTF \lTmpaClist {\prgReturn{Empty}} {\prgReturn{NonEmpty}}
\end{demohigh}
\end{function}

\begin{function}{\clistIfIn,\clistIfInT,\clistIfInF,\clistIfInTF}
\begin{syntax}
\cs{clistIfIn} \Arg{comma list} \Arg{item}
\cs{clistIfInT} \Arg{comma list} \Arg{item} \Arg{true code}
\cs{clistIfInF} \Arg{comma list} \Arg{item} \Arg{false code}
\cs{clistIfInTF} \Arg{comma list} \Arg{item} \Arg{true code} \Arg{false code}
\end{syntax}
Tests if the \meta{item} is present in the \meta{comma list}.
In the case of an \texttt{n}-type \meta{comma list}, the usual rules
of space trimming and brace stripping apply. For example
\begin{demohigh}
\clistIfInTF { a , {b}  , {b} , c } {b} {\prgReturn{Yes}} {\prgReturn{No}}
\clistIfInTF { a , {b}  , {b} , c } {d} {\prgReturn{Yes}} {\prgReturn{No}}
\end{demohigh}
%\begin{texnote}
%The function may fail if the \meta{item} contains \verb|{|, \verb|}|, or \verb|#|
%(assuming the usual \TeX{} category codes apply).
%\end{texnote}
\end{function}

\begin{function}{\clistVarIfIn,\clistVarIfInT,\clistVarIfInF,\clistVarIfInTF}
\begin{syntax}
\cs{clistVarIfIn} \meta{comma list} \Arg{item}
\cs{clistVarIfInT} \meta{comma list} \Arg{item} \Arg{true code}
\cs{clistVarIfInF} \meta{comma list} \Arg{item} \Arg{false code}
\cs{clistVarIfInTF} \meta{comma list} \Arg{item} \Arg{true code} \Arg{false code}
\end{syntax}
Tests if the \meta{item} is present in the \meta{comma list}.
In the case of an \texttt{n}-type \meta{comma list}, the usual rules
of space trimming and brace stripping apply.
\begin{demohigh}
\clistSet \lTmpaClist {one,two}
\clistVarIfInTF \lTmpaClist {one} {\prgReturn{Yes}} {\prgReturn{No}}
\clistVarIfInTF \lTmpaClist {three} {\prgReturn{Yes}} {\prgReturn{No}}
\end{demohigh}
%\begin{texnote}
%The function may fail if the \meta{item} contains \verb|{|, \verb|}|, or \verb|#|
%(assuming the usual \TeX{} category codes apply).
%\end{texnote}
\end{function}

\end{document}
