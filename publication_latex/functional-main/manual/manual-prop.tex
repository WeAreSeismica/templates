% -*- coding: utf-8 -*-
% !TEX program = lualatex
\documentclass[oneside]{book}

% -*- coding: utf-8 -*-
% !TEX program = lualatex

\usepackage[a4paper,margin=2.5cm]{geometry}

\newcommand*{\myversion}{2022H}
\newcommand*{\mydate}{Version \myversion\ (\the\year-\mylpad\month-\mylpad\day)}
\newcommand*{\mylpad}[1]{\ifnum#1<10 0\the#1\else\the#1\fi}

\setlength{\parindent}{0pt}
\setlength{\parskip}{4pt plus 1pt minus 1pt}

\usepackage{codehigh}

\colorlet{highback}{blue9}
%\CodeHigh{lite}
\CodeHigh{language=latex/latex2,style/main=highback,style/code=highback}
\NewCodeHighEnv{code}{style/main=gray9,style/code=gray9}
\NewCodeHighEnv{demo}{style/main=gray9,style/code=gray9,demo}

\usepackage{enumitem}

\NewDocumentCommand\MySubScript{m}{$_{#1}$}

\ExplSyntaxOn
\NewDocumentCommand\PrintVarList{m}{
  \clist_set:Nn \l_tmpa_clist {#1}
  \clist_map_inline:Nn \l_tmpa_clist
    {
      \token_to_str:N ##1 ~
    }
}
\NewDocumentCommand\RelaceChacters{m}{
  \tl_set:Nn \lTmpaTl {#1}
  \regex_replace_once:nnN { \_ } { \c{MySubScript} } \lTmpaTl
}
\NewDocumentCommand\RelaceUnderScoreAll{m}{
  \tl_set:Nn \lTmpaTl {#1}
  \regex_replace_all:nnN { \_ } { \c{_} } \lTmpaTl
  \lTmpaTl
}
\ExplSyntaxOff

\NewDocumentEnvironment{variable}{om}{
  \vspace{5pt}
  \begin{minipage}{\linewidth}
  \hrule\vspace{4pt}\obeylines%
  \begingroup
  \ttfamily\bfseries\color{azure3}
  \PrintVarList{#2}
  \endgroup
  \par\vspace{4pt}\hrule
  \end{minipage}\par\nopagebreak\vspace{4pt}
}{%
  \vspace{5pt}%
}

\NewDocumentEnvironment{function}{om}{
  \vspace{5pt}%
}{\vspace{5pt}}

\NewDocumentEnvironment{syntax}{}{%
  \begin{minipage}{\linewidth}
  \hrule\vspace{4pt}\obeylines%
}{%
  \par\vspace{4pt}\hrule
  \end{minipage}\par\nopagebreak\vspace{4pt}
}

\NewDocumentEnvironment{texnote}{}{}{}

\NewDocumentCommand\cs{O{}m}{%
  \texttt{\bfseries\color{purple3}\cBackslashStr\RelaceUnderScoreAll{#2}}%
}
\NewDocumentCommand\meta{m}{%
  \RelaceChacters{#1}%
  \textsl{$\langle$\ignorespaces\lTmpaTl\unskip$\rangle$}%
}
\NewDocumentCommand\Arg{m}{%
  \RelaceChacters{#1}%
  \texttt{\{}\textsl{$\langle$\ignorespaces\lTmpaTl\unskip$\rangle$}\texttt{\}}%
}

\NewDocumentCommand\pkg{m}{\textsf{#1}}

\NewDocumentCommand\nan{}{\texttt{NaN}}
\NewDocumentCommand\enquote{m}{``#1''}

\let\tn=\cs

\RenewDocumentCommand\emph{m}{%
  \underline{\textsl{#1}}%
}

\newenvironment{l3regex-syntax}
  {\begin{itemize}\def\\{\char`\\}\def\makelabel##1{\hss\llap{\ttfamily##1}}}
  {\end{itemize}}

\usepackage{shortvrb}

\NewDocumentCommand \MyMakeShortVerb {} {%
  \MakeShortVerb \|%
  \MakeShortVerb \"%
}
\NewDocumentCommand \MyDeleteShortVerb {} {%
  \DeleteShortVerb \|%
  \DeleteShortVerb \"%
}

\AtBeginDocument{\MyMakeShortVerb}
\AtEndDocument{\MyDeleteShortVerb}
\AddToHook{env/demohigh/before}{\MyDeleteShortVerb}
\AddToHook{env/demohigh/after}{\MyMakeShortVerb}

\usepackage{hologo}
\providecommand*\LuaTeX{\hologo{LuaTeX}}
\providecommand*\pdfTeX{\hologo{pdfTeX}}
\providecommand*\XeTeX{\hologo{XeTeX}}

\usepackage{amsmath}

\usepackage{hyperref}
\hypersetup{
  colorlinks=true,
  urlcolor=blue3,
  linkcolor=blue3,
}

\usepackage{functional}
%\Functional{scoping=false,tracing=true}

\CodeHigh{lite}

\begin{document}

\chapter{Property Lists (\texttt{Prop})}

\LaTeX3 implements a \enquote{property list} data type, which contain
an unordered list of entries each of which consists of a \meta{key} and
an associated \meta{value}. The \meta{key} and \meta{value} may both
be any \meta{balanced text}, the \meta{key} is processed using
\cs{tlToStr}, meaning that category codes are ignored. It is possible to
map functions to property lists such that the function is applied to every
key--value pair within the list.

Each entry in a property list must have a unique \meta{key}: if an entry is
added to a property list which already contains the \meta{key} then the new
entry overwrites the existing one. The \meta{keys} are compared on a
string basis, using the same method as \cs{strIfEq}.

%Property lists are intended for storing key-based information for use within
%code.  This is in contrast to key--value lists, which are a form of
%\emph{input} parsed by the \pkg{l3keys} module.

\section{Constant and Scratch Sequences}

\begin{variable}{\cEmptyProp}
Constant that is always empty.
\end{variable}

\begin{variable}{\lTmpaProp,\lTmpbProp,\lTmpcProp,\lTmpiProp,\lTmpjProp,\lTmpkProp}
Scratch property lists for local assignment. These are never used by
the \verb!functional! package, and so are safe for use with any
function. However, they may be overwritten by other
code and so should only be used for short-term storage.
\end{variable}

\begin{variable}{\gTmpaProp,\gTmpbProp,\gTmpcProp,\gTmpiProp,\gTmpjProp,\gTmpkProp}
Scratch property lists for global assignment. These are never used by
the \verb!functional! package, and so are safe for use with any
function. However, they may be overwritten by other
code and so should only be used for short-term storage.
\end{variable}

\section{Creating and Using Property Lists}

\begin{function}{\propNew}
\begin{syntax}
\cs{propNew} \meta{property list}
\end{syntax}
Creates a new \meta{property list} or raises an error if the name is
already taken. The declaration is global. The \meta{property list}
initially contains no entries.
\begin{codehigh}
\propNew \lFooSomeProp
\end{codehigh}
\end{function}

\begin{function}{\propConstFromKeyval}
\begin{syntax}
\cs{propConstFromKeyval} \meta{prop var}
~ \{
~ ~ \meta{key1} \verb|=| \meta{value1} \verb|,|
~ ~ \meta{key2} \verb|=| \meta{value2} \verb|,| $\cdots$
~ \}
\end{syntax}
Creates a new constant \meta{prop var} or raises an error if the
name is already taken. The \meta{prop var} is set globally to
contain key--value pairs given in the second argument, processed in
the way described for \cs{propSetFromKeyval}.  If duplicate
keys appear only the last of the values is kept.
This function correctly detects the \verb|=| and \verb|,| signs provided they
have the standard category code $12$ or they are active.
\begin{codehigh}
\propConstFromKeyval \cFooSomeProp {key1=one,key2=two,key3=three}
\end{codehigh}
%Notice that in contrast to most keyval lists (\emph{e.g.} those in
%\pkg{l3keys}), each key here \emph{must} be followed with an \texttt{=} sign.
\end{function}

\begin{function}{\propToKeyval}
\begin{syntax}
\cs{propToKeyval} \meta{property list}
\end{syntax}
Returns the \meta{property list} in a key--value notation. Keep in mind
that a \meta{property list} is \emph{unordered}, while key--value interfaces
don't necessarily are, so this can't be used for arbitrary interfaces.
%\begin{texnote}
%The result is returned within the \tn{unexpanded} primitive
%(\cs{exp_not:n}), which means that the key--value list does not expand
%further when appearing in an \texttt{x}-type or \texttt{e}-type argument expansion.
%It also needs exactly two steps of expansion.
%\end{texnote}
\begin{codehigh}
\propToKeyval \lTmpaProp
\end{codehigh}
\end{function}

\section{Viewing Property Lists}

\begin{function}{\propVarLog}
\begin{syntax}
\cs{propVarLog} \meta{property list}
\end{syntax}
Writes the entries in the \meta{property list} in the log file.
\begin{codehigh}
\propVarLog \lTmpaProp
\end{codehigh}
\end{function}

\begin{function}{\propVarShow}
\begin{syntax}
\cs{propVarShow} \meta{property list}
\end{syntax}
Displays the entries in the \meta{property list} in the terminal.
\begin{codehigh}
\propVarShow \lTmpaProp
\end{codehigh}
\end{function}

\section{Setting Property Lists}

\begin{function}{\propSetFromKeyval}
\begin{syntax}
\cs{propSetFromKeyval} \meta{prop var}
\{
~ \meta{key1} \verb|=| \meta{value1} \verb|,|
~ \meta{key2} \verb|=| \meta{value2} \verb|,| $\cdots$
\}
\end{syntax}
Sets \meta{prop var} to contain key--value pairs given in the second
argument.  If duplicate keys appear only the last of the values is kept.

Spaces are trimmed around every \meta{key} and every \meta{value},
and if the result of trimming spaces consists of a single brace
group then a set of outer braces is removed.  This enables both the
\meta{key} and the \meta{value} to contain spaces, commas or equal
signs.  The \meta{key} is then processed by \cs{tlToStr}.
This function correctly detects the \verb|=| and \verb|,| signs provided they
have the standard category code $12$ or they are active.
\begin{codehigh}
\propSetFromKeyval \lTmpaProp {key1=one,key2=two}
\end{codehigh}
%Notice that in contrast to most keyval lists (\emph{e.g.} those in
%\pkg{l3keys}), each key here \emph{must} be followed with an \texttt{=} sign.
\end{function}

\begin{function}{\propSetEq}
\begin{syntax}
\cs{propSetEq} \meta{property list_1} \meta{property list_2}
\end{syntax}
Sets the content of \meta{property list_1} equal to that of
\meta{property list_2}.
\begin{codehigh}
\propSetFromKeyval \lTmpaProp {key1=one,key2=two,key3=three}
\propSetEq \lTmpbProp \lTmpaProp
\propVarLog \lTmpbProp
\end{codehigh}
\end{function}

\begin{function}{\propClear}
\begin{syntax}
\cs{propClear} \meta{property list}
\end{syntax}
Clears all entries from the \meta{property list}.
\begin{codehigh}
\propClear \lTmpaProp
\end{codehigh}
\end{function}

\begin{function}{\propClearNew}
\begin{syntax}
\cs{propClearNew} \meta{property list}
\end{syntax}
Ensures that the \meta{property list} exists globally by applying \cs{propNew}
if necessary, then applies \cs{propClear} to leave the list empty.
\begin{codehigh}
\propClearNew \lFooSomeProp
\end{codehigh}
\end{function}

\begin{function}{\propConcat}
\begin{syntax}
\cs{propConcat} \meta{prop var_1} \meta{prop var_2} \meta{prop var_3}
\end{syntax}
Combines the key--value pairs of \meta{prop var_2} and
\meta{prop var_3}, and saves the result in \meta{prop var_1}.  If a
key appears in both \meta{prop var_2} and \meta{prop var_3} then the
last value, namely the value in \meta{prop var_3} is kept.
\begin{codehigh}
\propSetFromKeyval \lTmpbProp {key1=one,key2=two}
\propSetFromKeyval \lTmpcProp {key3=three,key4=four}
\propConcat \lTmpaProp \lTmpbProp \lTmpcProp
\propVarLog \lTmpaProp
\end{codehigh}
\end{function}

\begin{function}{\propPut}
\begin{syntax}
\cs{propPut} \meta{property list} \Arg{key} \Arg{value}
\end{syntax}
Adds an entry to the \meta{property list} which may be accessed
using the \meta{key} and which has \meta{value}. If the \meta{key}
is already present in the \meta{property list}, the existing entry
is overwritten by the new \meta{value}. Both the \meta{key} and
\meta{value} may contain any \meta{balanced text}. The \meta{key} is
stored after processing with \cs{tlToStr}, meaning that category
codes are ignored.
\begin{codehigh}
\propSetFromKeyval \lTmpaProp {key1=one,key2=two}
\propPut \lTmpaProp {key1} {newone}
\propVarLog \lTmpaProp
\end{codehigh}
\end{function}

\begin{function}{\propPutIfNew}
\begin{syntax}
\cs{propPutIfNew} \meta{property list} \Arg{key} \Arg{value}
\end{syntax}
If the \meta{key} is present in the \meta{property list} then no
action is taken. Otherwise, a new entry is added as described for
\cs{propPut}.
\begin{codehigh}
\propSetFromKeyval \lTmpaProp {key1=one,key2=two}
\propPutIfNew \lTmpaProp {key1} {newone}
\propVarLog \lTmpaProp
\end{codehigh}
\end{function}

\begin{function}{\propPutFromKeyval}
\begin{syntax}
\cs{propPutFromKeyval} \meta{prop var}
\{
~ \meta{key1} \verb|=| \meta{value1} \verb|,|
~ \meta{key2} \verb|=| \meta{value2} \verb|,| $\cdots$
\}
\end{syntax}
Updates the \meta{prop var} by adding entries for each key--value
pair given in the second argument.  The addition is done through
\cs{propPut}, hence if the \meta{prop var} already contains
some of the keys, the corresponding values are discarded and
replaced by those given in the key--value list.  If duplicate keys
appear in the key--value list then only the last of the values is kept.
\begin{codehigh}
\propSetFromKeyval \lTmpaProp {key1=one,key2=two}
\propPutFromKeyval \lTmpaProp {key1=newone,key3=three}
\propVarLog \lTmpaProp
\end{codehigh}
%The function is equivalent to storing the key--value pairs in a
%temporary property variable using \cs{propSetFromKeyval}, then
%combining \meta{prop var} with the temporary variable using
%\cs{propConcat}.  In particular, the \meta{keys} and
%\meta{values} are space-trimmed and unbraced as described in
%\cs{propSetFromKeyval}. This function correctly detects
%the \verb|=| and \verb|,| signs provided they
%have the standard category code $12$ or they are active.
\end{function}

\begin{function}{\propVarRemove}
\begin{syntax}
\cs{propVarRemove} \meta{property list} \Arg{key}
\end{syntax}
Removes the entry listed under \meta{key} from the
\meta{property list}.  If the \meta{key} is
not found in the \meta{property list} no change occurs,
\emph{i.e} there is no need to test for the existence of a key before
deleting it.
\begin{codehigh}
\propSetFromKeyval \lTmpaProp {key1=one,key2=two,key3=three}
\propVarRemove \lTmpaProp {key2}
\propVarLog \lTmpaProp
\end{codehigh}
\end{function}

\section{Recovering Values from Property Lists}

\begin{function}{\propVarCount}
\begin{syntax}
\cs{propVarCount} \meta{property list}
\end{syntax}
Returns the number of key--value pairs in the \meta{property list}
as an \meta{integer denotation}.
\begin{demohigh}
\propSetFromKeyval \lTmpaProp {key1=one,key2=two,key3=three}
\propVarCount \lTmpaProp
\end{demohigh}
\end{function}

\begin{function}{\propVarItem}
\begin{syntax}
\cs{propVarItem} \meta{property list} \Arg{key}
\end{syntax}
Returns the \meta{value} corresponding to the \meta{key} in
the \meta{property list}. If the \meta{key} is missing,
nothing is returned.
\begin{demohigh}
\propSetFromKeyval \lTmpaProp {key1=one,key2=two,key3=three}
\tlSet \lTmpaTl {\propVarItem \lTmpaProp {key2}}
\tlUse \lTmpaTl
\end{demohigh}
%\begin{texnote}
%This function is slower than the non-expandable analogue
%\cs{prop_get:NnN}.
%The result is returned within the \tn{unexpanded}
%primitive (\cs{exp_not:n}), which means that the \meta{value}
%does not expand further when appearing in an \texttt{x}-type
%or \texttt{e}-type argument expansion.
%\end{texnote}
\end{function}

\begin{function}{\propGet}
\begin{syntax}
\cs{propGet} \meta{property list} \Arg{key} \meta{token list variable}
\end{syntax}
Recovers the \meta{value} stored with \meta{key} from the \meta{property list},
and places this in the \meta{token list variable}.
If the \meta{key} is not found in the
\meta{property list} then the \meta{token list variable} is set
to the special marker \cs{qNoValue}.
The assignment of the \meta{token list variable} is local.
\begin{demohigh}
\propSetFromKeyval \lTmpaProp {key1=one,key2=two,key3=three}
\propGet \lTmpaProp {key2} \lTmpaTl
\tlUse \lTmpaTl
\end{demohigh}
\end{function}

\begin{function}{\propGetT,\propGetF,\propGetTF}
\begin{syntax}
\cs{propGetT} \meta{property list} \Arg{key} \meta{token list variable} \Arg{true code}
\cs{propGetF} \meta{property list} \Arg{key} \meta{token list variable} \Arg{false code}
\cs{propGetTF} \meta{property list} \Arg{key} \meta{token list variable} \Arg{true code} \Arg{false code}
\end{syntax}
If the \meta{key} is not present in the \meta{property list}, leaves
the \meta{false code} in the input stream.  The value of the
\meta{token list variable} is not defined in this case and should
not be relied upon.  If the \meta{key} is present in the
\meta{property list}, stores the corresponding \meta{value} in the
\meta{token list variable} without removing it from the
\meta{property list}, then leaves the \meta{true code} in the input
stream.  The \meta{token list variable} is assigned locally.
\begin{demohigh}
\propSetFromKeyval \lTmpaProp {key1=one,key2=two,key3=three}
\propGetTF \lTmpaProp {key2} \lTmpaTl {\prgReturn{Yes}} {\prgReturn{No}}
\end{demohigh}
\end{function}

\begin{function}{\propPop}
\begin{syntax}
\cs{propPop} \meta{property list} \Arg{key} \meta{token list variable}
\end{syntax}
Recovers the \meta{value} stored with \meta{key} from the \meta{property list},
and places this in the \meta{token list variable}.
If the \meta{key} is not found in the
\meta{property list} then the \meta{token list variable} is set
to the special marker \cs{qNoValue}.
The \meta{key} and \meta{value} are then deleted from the property list.
The assignment of the \meta{token list variable} is local.
\begin{demohigh}
\propSetFromKeyval \lTmpaProp {key1=one,key2=two,key3=three}
\propPop \lTmpaProp {key2} \lTmpaTl
Pop: \tlUse \lTmpaTl.
Count: \propVarCount \lTmpaProp.
\end{demohigh}
\end{function}

\begin{function}{\propPopT,\propPopF,\propPopTF}
\begin{syntax}
\cs{propPopT} \meta{property list} \Arg{key} \meta{token list variable} \Arg{true code}
\cs{propPopF} \meta{property list} \Arg{key} \meta{token list variable} \Arg{false code}
\cs{propPopTF} \meta{property list} \Arg{key} \meta{token list variable} \Arg{true code} \Arg{false code}
\end{syntax}
If the \meta{key} is not present in the \meta{property list}, leaves
the \meta{false code} in the input stream.  The value of the
\meta{token list variable} is not defined in this case and should
not be relied upon.  If the \meta{key} is present in
the \meta{property list}, pops the corresponding \meta{value}
in the \meta{token list variable}, \emph{i.e.} removes the item from
The \meta{token list variable} is assigned locally.
\begin{demohigh}
\propSetFromKeyval \lTmpaProp {key1=one,key2=two,key3=three}
\propPopTF \lTmpaProp {key2} \lTmpaTl {\prgReturn{Yes}} {\prgReturn{No}}
\end{demohigh}
\end{function}

\section{Mapping over property lists}

%All mappings are done at the current group level, \emph{i.e.} any
%local assignments made by the \meta{function} or \meta{code} discussed
%below remain in effect after the loop.

%\begin{function}{\propVarMapFunction}
%\begin{syntax}
%\cs{propVarMapFunction} \meta{property list} \meta{function}
%\end{syntax}
%Applies \meta{function} to every \meta{entry} stored in the
%\meta{property list}. The \meta{function} receives two arguments for
%each iteration: the \meta{key} and associated \meta{value}.
%The order in which \meta{entries} are returned is not defined and
%should not be relied upon.
%To pass further arguments to the \meta{function}, see
%\cs{prop_map_tokens:Nn}.
%\end{function}

\begin{function}{\propVarMapInline}
\begin{syntax}
\cs{propVarMapInline} \meta{property list} \Arg{inline function}
\end{syntax}
Applies \meta{inline function} to every \meta{entry} stored
within the \meta{property list}. The \meta{inline function} should
consist of code which receives the \meta{key} as \verb|#1| and the
\meta{value} as \verb|#2|.
The order in which \meta{entries} are returned is not defined and
should not be relied upon.
\begin{demohigh}
\IgnoreSpacesOn
\propSetFromKeyval \lTmpkProp {key1=one,key2=two,key3=three}
\tlClear \lTmpaTl
\propVarMapInline \lTmpkProp {
  \tlPutRight \lTmpaTl {(#1=#2)}
}
\tlUse \lTmpaTl
\IgnoreSpacesOff
\end{demohigh}
\end{function}

%\begin{function}{\propVarMapTokens}
%\begin{syntax}
%\cs{propVarMapTokens} \meta{property list} \Arg{code}
%\end{syntax}
%Analogue of \cs{propVarMapFunction} which maps several tokens
%instead of a single function.  The \meta{code} receives each
%key--value pair in the \meta{property list} as two trailing brace
%groups. For instance,
%\begin{verbatim}
%\propVarMapTokens \lMyProp { \strIfEqT { mykey } }
%\end{verbatim}
%expands to the value corresponding to \texttt{mykey}: for each
%pair in \verb|\lMyProp| the function \cs{strIfEqT} receives
%\texttt{mykey}, the \meta{key} and the \meta{value} as its three
%arguments. For that specific task, \cs{propVarItem} is faster.
%\end{function}

%\begin{function}{\propMapBreak}
%\begin{syntax}
%\cs{propMapBreak}
%\end{syntax}
%Used to terminate a prop map function before all
%entries in the \meta{property list} have been processed. This
%normally takes place within a conditional statement, for example
%\begin{verbatim}
%\prop_map_inline:Nn \l_my_prop
%{
%\str_if_eq:nnTF { #1 } { bingo }
%{ \prop_map_break: }
%{
%Do something useful
%}
%}
%\end{verbatim}
%Use outside of a prop map scenario leads to low
%level \TeX{} errors.
%\begin{texnote}
%When the mapping is broken, additional tokens may be inserted
%before further items are taken
%from the input stream. This depends on the design of the mapping
%function.
%\end{texnote}
%\end{function}

%\begin{function}{\propMapBreakDo}
%\begin{syntax}
%\cs{propMapBreakDo} \Arg{code}
%\end{syntax}
%Used to terminate a prop map function before all
%entries in the \meta{property list} have been processed, inserting
%the \meta{code} after the mapping has ended. This
%normally takes place within a conditional statement, for example
%\begin{verbatim}
%\prop_map_inline:Nn \l_my_prop
%{
%\str_if_eq:nnTF { #1 } { bingo }
%{ \prop_map_break:n { <code> } }
%{
%Do something useful
%}
%}
%\end{verbatim}
%Use outside of a prop map scenario leads to low
%level \TeX{} errors.
%\begin{texnote}
%When the mapping is broken, additional tokens may be inserted
%before the \meta{code} is
%inserted into the input stream.
%This depends on the design of the mapping function.
%\end{texnote}
%\end{function}

\section{Property List Conditionals}

\begin{function}{\propIfExist,\propIfExistT,\propIfExistF,\propIfExistTF}
\begin{syntax}
\cs{propIfExist} \meta{property list}
\cs{propIfExistT} \meta{property list} \Arg{true code}
\cs{propIfExistF} \meta{property list} \Arg{false code}
\cs{propIfExistTF} \meta{property list} \Arg{true code} \Arg{false code}
\end{syntax}
Tests whether the \meta{property list} is currently defined.  This does not
check that the \meta{property list} really is a property list variable.
\begin{demohigh}
\propIfExistTF \lTmpaProp {\prgReturn{Yes}} {\prgReturn{No}}
\propIfExistTF \lFooUndefinedProp {\prgReturn{Yes}} {\prgReturn{No}}
\end{demohigh}
\end{function}

\begin{function}{\propVarIfEmpty,\propVarIfEmptyT,\propVarIfEmptyF,\propVarIfEmptyTF}
\begin{syntax}
\cs{propVarIfEmpty} \meta{property list}
\cs{propVarIfEmptyT} \meta{property list} \Arg{true code}
\cs{propVarIfEmptyF} \meta{property list} \Arg{false code}
\cs{propVarIfEmptyTF} \meta{property list} \Arg{true code} \Arg{false code}
\end{syntax}
Tests if the \meta{property list} is empty (containing no entries).
\begin{demohigh}
\propSetFromKeyval \lTmpaProp {key1=one,key2=two}
\propVarIfEmptyTF \lTmpaProp {\prgReturn{Empty}} {\prgReturn{NonEmpty}}
\propClear \lTmpaProp
\propVarIfEmptyTF \lTmpaProp {\prgReturn{Empty}} {\prgReturn{NonEmpty}}
\end{demohigh}
\end{function}

\begin{function}{\propVarIfIn,\propVarIfInT,\propVarIfInF,\propVarIfInTF}
\begin{syntax}
\cs{propVarIfIn} \meta{property list} \Arg{key}
\cs{propVarIfInT} \meta{property list} \Arg{key} \Arg{true code}
\cs{propVarIfInF} \meta{property list} \Arg{key} \Arg{false code}
\cs{propVarIfInTF} \meta{property list} \Arg{key} \Arg{true code} \Arg{false code}
\end{syntax}
Tests if the \meta{key} is present in the \meta{property list},
making the comparison using the method described by \cs{strIfEqTF}.
\begin{demohigh}
\propSetFromKeyval \lTmpaProp {key1=one,key2=two}
\propVarIfInTF \lTmpaProp {key1} {\prgReturn{Yes}} {\prgReturn{Not}}
\propVarIfInTF \lTmpaProp {key3} {\prgReturn{Yes}} {\prgReturn{Not}}
\end{demohigh}
%\begin{texnote}
%This function iterates through every key--value pair in the
%\meta{property list} and is therefore slower than using \cs{propGetTF}.
%\end{texnote}
\end{function}

\end{document}
