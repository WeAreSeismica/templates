% -*- coding: utf-8 -*-
% !TEX program = lualatex
\documentclass[oneside]{book}

% -*- coding: utf-8 -*-
% !TEX program = lualatex

\usepackage[a4paper,margin=2.5cm]{geometry}

\newcommand*{\myversion}{2022H}
\newcommand*{\mydate}{Version \myversion\ (\the\year-\mylpad\month-\mylpad\day)}
\newcommand*{\mylpad}[1]{\ifnum#1<10 0\the#1\else\the#1\fi}

\setlength{\parindent}{0pt}
\setlength{\parskip}{4pt plus 1pt minus 1pt}

\usepackage{codehigh}

\colorlet{highback}{blue9}
%\CodeHigh{lite}
\CodeHigh{language=latex/latex2,style/main=highback,style/code=highback}
\NewCodeHighEnv{code}{style/main=gray9,style/code=gray9}
\NewCodeHighEnv{demo}{style/main=gray9,style/code=gray9,demo}

\usepackage{enumitem}

\NewDocumentCommand\MySubScript{m}{$_{#1}$}

\ExplSyntaxOn
\NewDocumentCommand\PrintVarList{m}{
  \clist_set:Nn \l_tmpa_clist {#1}
  \clist_map_inline:Nn \l_tmpa_clist
    {
      \token_to_str:N ##1 ~
    }
}
\NewDocumentCommand\RelaceChacters{m}{
  \tl_set:Nn \lTmpaTl {#1}
  \regex_replace_once:nnN { \_ } { \c{MySubScript} } \lTmpaTl
}
\NewDocumentCommand\RelaceUnderScoreAll{m}{
  \tl_set:Nn \lTmpaTl {#1}
  \regex_replace_all:nnN { \_ } { \c{_} } \lTmpaTl
  \lTmpaTl
}
\ExplSyntaxOff

\NewDocumentEnvironment{variable}{om}{
  \vspace{5pt}
  \begin{minipage}{\linewidth}
  \hrule\vspace{4pt}\obeylines%
  \begingroup
  \ttfamily\bfseries\color{azure3}
  \PrintVarList{#2}
  \endgroup
  \par\vspace{4pt}\hrule
  \end{minipage}\par\nopagebreak\vspace{4pt}
}{%
  \vspace{5pt}%
}

\NewDocumentEnvironment{function}{om}{
  \vspace{5pt}%
}{\vspace{5pt}}

\NewDocumentEnvironment{syntax}{}{%
  \begin{minipage}{\linewidth}
  \hrule\vspace{4pt}\obeylines%
}{%
  \par\vspace{4pt}\hrule
  \end{minipage}\par\nopagebreak\vspace{4pt}
}

\NewDocumentEnvironment{texnote}{}{}{}

\NewDocumentCommand\cs{O{}m}{%
  \texttt{\bfseries\color{purple3}\cBackslashStr\RelaceUnderScoreAll{#2}}%
}
\NewDocumentCommand\meta{m}{%
  \RelaceChacters{#1}%
  \textsl{$\langle$\ignorespaces\lTmpaTl\unskip$\rangle$}%
}
\NewDocumentCommand\Arg{m}{%
  \RelaceChacters{#1}%
  \texttt{\{}\textsl{$\langle$\ignorespaces\lTmpaTl\unskip$\rangle$}\texttt{\}}%
}

\NewDocumentCommand\pkg{m}{\textsf{#1}}

\NewDocumentCommand\nan{}{\texttt{NaN}}
\NewDocumentCommand\enquote{m}{``#1''}

\let\tn=\cs

\RenewDocumentCommand\emph{m}{%
  \underline{\textsl{#1}}%
}

\newenvironment{l3regex-syntax}
  {\begin{itemize}\def\\{\char`\\}\def\makelabel##1{\hss\llap{\ttfamily##1}}}
  {\end{itemize}}

\usepackage{shortvrb}

\NewDocumentCommand \MyMakeShortVerb {} {%
  \MakeShortVerb \|%
  \MakeShortVerb \"%
}
\NewDocumentCommand \MyDeleteShortVerb {} {%
  \DeleteShortVerb \|%
  \DeleteShortVerb \"%
}

\AtBeginDocument{\MyMakeShortVerb}
\AtEndDocument{\MyDeleteShortVerb}
\AddToHook{env/demohigh/before}{\MyDeleteShortVerb}
\AddToHook{env/demohigh/after}{\MyMakeShortVerb}

\usepackage{hologo}
\providecommand*\LuaTeX{\hologo{LuaTeX}}
\providecommand*\pdfTeX{\hologo{pdfTeX}}
\providecommand*\XeTeX{\hologo{XeTeX}}

\usepackage{amsmath}

\usepackage{hyperref}
\hypersetup{
  colorlinks=true,
  urlcolor=blue3,
  linkcolor=blue3,
}

\usepackage{functional}
%\Functional{scoping=false,tracing=true}

\CodeHigh{lite}

\begin{document}

\chapter{Dimensions (\texttt{Dim})}

\section{Constant and Scratch Dimensions}

\begin{variable}{\cMaxDim}
The maximum value that can be stored as a dimension.  This can also
be used as a component of a skip.
\end{variable}

\begin{variable}{\cZeroDim}
A zero length as a dimension.  This can also be used as a component
of a skip.
\end{variable}

\begin{variable}{\lTmpaDim,\lTmpbDim,\lTmpcDim,\lTmpiDim,\lTmpjDim,\lTmpkDim}
Scratch dimensions for local assignment. These are never used by
the \verb!functional! package, and so are safe for use with any
function. However, they may be overwritten by other
code and so should only be used for short-term storage.
\end{variable}

\begin{variable}{\gTmpaDim,\gTmpbDim,\gTmpcDim,\gTmpiDim,\gTmpjDim,\gTmpkDim}
Scratch dimensions for global assignment. These are never used by
the \verb!functional! package, and so are safe for use with any
function. However, they may be overwritten by other
code and so should only be used for short-term storage.
\end{variable}

\section{Dimension Expressions}

\begin{function}{\dimEval}
\begin{syntax}
\cs{dimEval} \Arg{dimension expression}
\end{syntax}
Evaluates the \meta{dimension expression}, expanding any
dimensions and token list variables within the \meta{expression}
to their content (without requiring \cs{dimUse}/\cs{tlUse})
and applying the standard mathematical rules. The result of the
calculation is returned as a \meta{dimension denotation}.
%This is expressed in points (\texttt{pt}), and requires suitable
%termination if used in a \TeX{}-style assignment as it is \emph{not}
%an \meta{internal dimension}.
For example
\begin{demohigh}
\dimEval {(1.2pt+3.4pt)/9}
\end{demohigh}
%\dimEval{(1.2pt+3.4pt)*(5.6-7.8)/9}
\end{function}

\begin{function}{\dimMathAdd}
\begin{syntax}
\cs{dimMathAdd} \Arg{dimexpr_1} \Arg{dimexpr_2}
\end{syntax}
Adds \Arg{dimexpr_1} and \Arg{dimexpr_2},
and returns the result. For example
\begin{demohigh}
\dimMathAdd {2.8pt} {3.7pt}
\dimMathAdd {3.8pt-1pt} {2.7pt+1pt}
\end{demohigh}
\end{function}

\begin{function}{\dimMathSub}
\begin{syntax}
\cs{dimMathSub} \Arg{dimexpr_1} \Arg{dimexpr_2}
\end{syntax}
Subtracts \Arg{dimexpr_2} from \Arg{dimexpr_1},
and returns the result. For example
\begin{demohigh}
\dimMathSub {2.8pt} {3.7pt}
\dimMathSub {3.8pt-1pt} {2.7pt+1pt}
\end{demohigh}
\end{function}

\begin{function}{\dimMathRatio}
\begin{syntax}
\cs{dimMathRatio} \Arg{dimexpr_1} \Arg{dimexpr_2}
\end{syntax}
Parses the two \meta{dimension expressions},
then calculates the ratio of the two and returns it.
The result is a ratio expression
between two integers, with all distances converted to scaled points.
For example
\begin{demohigh}
\dimMathRatio {5pt} {10pt}
\end{demohigh}
The returned value is suitable for use inside a \meta{dimension expression}
such as
\begin{codehigh}
\dimSet \lTmpaDim {10pt*\dimMathRatio{5pt}{10pt}}
\end{codehigh}
\end{function}

\begin{function}{\dimMathSign}
\begin{syntax}
\cs{dimMathSign} \Arg{dimexpr}
\end{syntax}
Evaluates the \meta{dimexpr} then returns $1$ or $0$ or $-1$
according to the sign of the result. For example
\begin{demohigh}
\dimMathSign {3.5pt}
\dimMathSign {-2.7pt}
\end{demohigh}
\end{function}

\begin{function}{\dimMathAbs}
\begin{syntax}
\cs{dimMathAbs} \Arg{dimexpr}
\end{syntax}
Converts the \meta{dimexpr} to its absolute value,
returning the result as a \meta{dimension denotation}.
For example
\begin{demohigh}
\dimMathAbs {3.5pt}
\dimMathAbs {-2.7pt}
\end{demohigh}
\end{function}

\begin{function}{\dimMathMax,\dimMathMin}
\begin{syntax}
\cs{dimMathMax} \Arg{dimexpr_1} \Arg{dimexpr_2}
\cs{dimMathMin} \Arg{dimexpr_1} \Arg{dimexpr_2}
\end{syntax}
Evaluates the two \meta{dimension expressions} and returns either the
maximum or minimum value as appropriate as a \meta{dimension denotation}.
For example
\begin{demohigh}
\dimMathMax {3.5pt} {-2.7pt}
\dimMathMin {3.5pt} {-2.7pt}
\end{demohigh}
\end{function}

\section{Creating and Using Dimensions}

\begin{function}{\dimNew}
\begin{syntax}
\cs{dimNew} \meta{dimension}
\end{syntax}
Creates a new \meta{dimension} or raises an error if the name is
already taken. The declaration is global. The \meta{dimension}
is initially equal to $0$\,pt.
\end{function}

\begin{function}{\dimConst}
\begin{syntax}
\cs{dimConst} \meta{dimension} \Arg{dimension expression}
\end{syntax}
Creates a new constant \meta{dimension} or raises an error if the
name is already taken. The value of the \meta{dimension} is set
globally to the \meta{dimension expression}. For example
\begin{demohigh}
\dimConst \cFooSomeDim {1cm}
\dimUse \cFooSomeDim
\end{demohigh}
\end{function}

\begin{function}{\dimUse}
\begin{syntax}
\cs{dimUse} \meta{dimension}
\end{syntax}
Recovers the content of a \meta{dimension} and returns the value.
An error is raised if the variable does not exist or if it is invalid.
\end{function}

\section{Viewing Dimensions}

\begin{function}{\dimLog}
\begin{syntax}
\cs{dimLog} \Arg{dimension expression}
\end{syntax}
Writes the result of evaluating the \meta{dimension expression}
in the log file. For example
\begin{codehigh}
\dimLog {\lFooSomeDim+1cm}
\end{codehigh}
\end{function}

\begin{function}{\dimVarLog}
\begin{syntax}
\cs{dimVarLog} \meta{dimension}
\end{syntax}
Writes the value of the \meta{dimension} in the log file. For example
\begin{codehigh}
\dimVarLog \lFooSomeDim
\end{codehigh}
\end{function}

\begin{function}{\dimShow}
\begin{syntax}
\cs{dimShow} \Arg{dimension expression}
\end{syntax}
Displays the result of evaluating the \meta{dimension expression}
on the terminal. For example
\begin{codehigh}
\dimShow {\lFooSomeDim+1cm}
\end{codehigh}
\end{function}

\begin{function}{\dimVarShow}
\begin{syntax}
\cs{dimVarShow} \meta{dimension}
\end{syntax}
Displays the value of the \meta{dimension} on the terminal. For example
\begin{codehigh}
\dimVarShow \lFooSomeDim
\end{codehigh}
\end{function}

\section{Setting Dimension Variables}

\begin{function}{\dimSet}
\begin{syntax}
\cs{dimSet} \meta{dimension} \Arg{dimension expression}
\end{syntax}
Sets \meta{dimension} to the value of \meta{dimension expression}, which
must evaluate to a length with units.
\end{function}

\begin{function}{\dimSetEq}
\begin{syntax}
\cs{dimSetEq} \meta{dimension_1} \meta{dimension_2}
\end{syntax}
Sets the content of \meta{dimension_1} equal to that of
\meta{dimension_2}. For example
\begin{demohigh}
\dimSet \lTmpaDim {10pt}
\dimSetEq \lTmpbDim \lTmpaDim
\dimUse \lTmpbDim
\end{demohigh}
\end{function}

\begin{function}{\dimZero}
\begin{syntax}
\cs{dimZero} \meta{dimension}
\end{syntax}
Sets \meta{dimension} to $0$\,pt. For example
\begin{demohigh}
\dimSet \lTmpaDim {1em}
\dimZero \lTmpaDim
\dimUse \lTmpaDim
\end{demohigh}
\end{function}

\begin{function}{\dimZeroNew}
\begin{syntax}
\cs{dimZeroNew} \meta{dimension}
\end{syntax}
Ensures that the \meta{dimension} exists globally by applying
\cs{dimNew} if necessary, then applies
\cs{dimZero} to set the \meta{dimension} to zero. For example
\begin{demohigh}
\dimZeroNew \lFooSomeDim
\dimUse \lFooSomeDim
\end{demohigh}
\end{function}

\begin{function}{\dimAdd}
\begin{syntax}
\cs{dimAdd} \meta{dimension} \Arg{dimension expression}
\end{syntax}
Adds the result of the \meta{dimension expression} to the current
content of the \meta{dimension}. For example
\begin{demohigh}
\dimSet \lTmpaDim {5.3pt}
\dimAdd \lTmpaDim {2.11pt}
\dimUse \lTmpaDim
\end{demohigh}
\end{function}

\begin{function}{\dimSub}
\begin{syntax}
\cs{dimSub} \meta{dimension} \Arg{dimension expression}
\end{syntax}
Subtracts the result of the \meta{dimension expression} from the
current content of the \meta{dimension}. For example
\begin{demohigh}
\dimSet \lTmpaDim {5.3pt}
\dimSub \lTmpaDim {2.11pt}
\dimUse \lTmpaDim
\end{demohigh}
\end{function}

\section{Dimension Step Functions}

%\begin{function}{\dimStepFunction}
%\begin{syntax}
%\cs{dimStepFunction} \Arg{initial value} \Arg{step} \Arg{final value} \meta{function}
%\end{syntax}
%This function first evaluates the \meta{initial value}, \meta{step}
%and \meta{final value}, all of which should be dimension expressions.
%The \meta{function} is then placed in front of each \meta{value}
%from the \meta{initial value} to the \meta{final value} in turn
%(using \meta{step} between each \meta{value}).  The \meta{step} must
%be non-zero.  If the \meta{step} is positive, the loop stops when
%the \meta{value} becomes larger than the \meta{final value}.  If the
%\meta{step} is negative, the loop stops when the \meta{value}
%becomes smaller than the \meta{final value}.  The \meta{function}
%should absorb one argument.
%\end{function}

\begin{function}{\dimStepInline}
\begin{syntax}
\cs{dimStepInline} \Arg{initial value} \Arg{step} \Arg{final value} \Arg{code}
\end{syntax}
This function first evaluates the \meta{initial value}, \meta{step}
and \meta{final value}, all of which should be dimension expressions.
Then for each \meta{value} from the \meta{initial value} to the
\meta{final value} in turn (using \meta{step} between each
\meta{value}), the \meta{code} is inserted into the input stream
with \verb|#1| replaced by the current \meta{value}.  Thus the
\meta{code} should define a function of one argument (\verb|#1|).
\begin{demohigh}
\IgnoreSpacesOn
\tlClear \lTmpaTl
\dimStepInline {1pt} {0.1pt} {1.5pt} {
  \tlPutRight \lTmpaTl {[#1]}
}
\tlUse \lTmpaTl
\IgnoreSpacesOff
\end{demohigh}
\end{function}

\begin{function}{\dimStepVariable}
\begin{syntax}
\cs{dimStepVariable} \Arg{initial value} \Arg{step} \Arg{final value} \meta{tl var} \Arg{code}
\end{syntax}
This function first evaluates the \meta{initial value}, \meta{step}
and \meta{final value}, all of which should be dimension expressions.
Then for each \meta{value} from the \meta{initial value} to the
\meta{final value} in turn (using \meta{step} between each
\meta{value}), the \meta{code} is inserted into the input stream,
with the \meta{tl var} defined as the current \meta{value}.  Thus
the \meta{code} should make use of the \meta{tl var}.
%For example
%\begin{demohigh}
%\IgnoreSpacesOn
%\tlClear\lTmpaTl
%\dimStepVariable{1pt}{0.1pt}{1.5pt}\lTmpiTl{
%  \tlPutRight\lTmpaTl{\expValue\lTmpiTl}
%  \tlPutRight\lTmpaTl{~}
%}
%\prgReturn{\expValue\lTmpaTl}
%\IgnoreSpacesOff
%\end{demohigh}
\end{function}

\section{Dimension Conditionals}

\begin{function}{\dimIfExist,\dimIfExistT,\dimIfExistF,\dimIfExistTF}
\begin{syntax}
\cs{dimIfExist} \meta{dimension}
\cs{dimIfExistT} \meta{dimension} \Arg{true code}
\cs{dimIfExistF} \meta{dimension} \Arg{false code}
\cs{dimIfExistTF} \meta{dimension} \Arg{true code} \Arg{false code}
\end{syntax}
Tests whether the \meta{dimension} is currently defined.  This does
not check that the \meta{dimension} really is a dimension variable.
For example
\begin{demohigh}
\dimIfExistTF \lTmpaDim {\prgReturn{Yes}} {\prgReturn{No}}
\dimIfExistTF \lFooUndefinedDim {\prgReturn{Yes}} {\prgReturn{No}}
\end{demohigh}
\end{function}

\begin{function}{\dimCompare,\dimCompareT,\dimCompareF,\dimCompareTF}
\begin{syntax}
\cs{dimCompare} \Arg{dimexpr_1} \meta{relation} \Arg{dimexpr_2}
\cs{dimCompareT} \Arg{dimexpr_1} \meta{relation} \Arg{dimexpr_2} \Arg{true code}
\cs{dimCompareF} \Arg{dimexpr_1} \meta{relation} \Arg{dimexpr_2} \Arg{false code}
\cs{dimCompareTF} \Arg{dimexpr_1} \meta{relation} \Arg{dimexpr_2} \Arg{true code} \Arg{false code}
\end{syntax}
This function first evaluates each of the \meta{dimension expressions}
as described for \cs{dimEval}. The two results are then
compared using the \meta{relation}:\par
{\centering
\begin{tabular}{ll}
Equal        & \verb|=| \\
Greater than & \verb|>| \\
Less than    & \verb|<| \\
\end{tabular}\par}
For example
\begin{demohigh}
\dimCompareTF {1pt} > {0.9999pt} {\prgReturn{Greater}} {\prgReturn{Less}}
\dimCompareTF {1pt} > {1.0001pt} {\prgReturn{Greater}} {\prgReturn{Less}}
\end{demohigh}
\end{function}

\section{Dimension Case Functions}

\begin{function}{\dimCase}
\begin{syntax}
\cs{dimCase} \Arg{test dimension expression}
~ ~ \verb|{|
~ ~ ~ ~ \Arg{dimexpr case_1} \Arg{code case_1}
~ ~ ~ ~  \Arg{dimexpr case_2} \Arg{code case_2}
~ ~ ~ ~ \ldots
~ ~ ~ ~ \Arg{dimexpr case_n} \Arg{code case_n}
~ ~ \verb|}|
\end{syntax}
This function evaluates the \meta{test dimension expression} and
compares this in turn to each of the
\meta{dimension expression cases}. If the two are equal then the
associated \meta{code} is left in the input stream
and other cases are discarded.
\end{function}

\begin{function}{\dimCaseT}
\begin{syntax}
\cs{dimCaseT} \Arg{test dimension expression}
~ ~ \verb|{|
~ ~ ~ ~ \Arg{dimexpr case_1} \Arg{code case_1}
~ ~ ~ ~  \Arg{dimexpr case_2} \Arg{code case_2}
~ ~ ~ ~ \ldots
~ ~ ~ ~ \Arg{dimexpr case_n} \Arg{code case_n}
~ ~ \verb|}|
~ ~ \Arg{true code}
\end{syntax}
This function evaluates the \meta{test dimension expression} and
compares this in turn to each of the
\meta{dimension expression cases}. If the two are equal then the
associated \meta{code} is left in the input stream
and other cases are discarded. If any of the
cases are matched, the \meta{true code} is also inserted into the
input stream (after the code for the appropriate case).
\end{function}

\begin{function}{\dimCaseF}
\begin{syntax}
\cs{dimCaseF} \Arg{test dimension expression}
~ ~ \verb|{|
~ ~ ~ ~ \Arg{dimexpr case_1} \Arg{code case_1}
~ ~ ~ ~  \Arg{dimexpr case_2} \Arg{code case_2}
~ ~ ~ ~ \ldots
~ ~ ~ ~ \Arg{dimexpr case_n} \Arg{code case_n}
~ ~ \verb|}|
~ ~ \Arg{false code}
\end{syntax}
This function evaluates the \meta{test dimension expression} and
compares this in turn to each of the
\meta{dimension expression cases}. If the two are equal then the
associated \meta{code} is left in the input stream
and other cases are discarded. If none of the cases
match then the \meta{false code} is inserted.
For example
\begin{demohigh}
\IgnoreSpacesOn
\dimSet \lTmpaDim {5pt}
\dimCaseF {2\lTmpaDim} {
  {5pt}     {\prgReturn{Small}}
  {4pt+6pt} {\prgReturn{Medium}}
  {-10pt}   {\prgReturn{Negative}}
}{
  \prgReturn {No Match}
}
\IgnoreSpacesOff
\end{demohigh}
\end{function}

\begin{function}{\dimCaseTF}
\begin{syntax}
\cs{dimCaseTF} \Arg{test dimension expression}
~ ~ \verb|{|
~ ~ ~ ~ \Arg{dimexpr case_1} \Arg{code case_1}
~ ~ ~ ~  \Arg{dimexpr case_2} \Arg{code case_2}
~ ~ ~ ~ \ldots
~ ~ ~ ~ \Arg{dimexpr case_n} \Arg{code case_n}
~ ~ \verb|}|
~ ~ \Arg{true code}
~ ~ \Arg{false code}
\end{syntax}
This function evaluates the \meta{test dimension expression} and
compares this in turn to each of the
\meta{dimension expression cases}. If the two are equal then the
associated \meta{code} is left in the input stream
and other cases are discarded. If any of the
cases are matched, the \meta{true code} is also inserted into the
input stream (after the code for the appropriate case), while if none
match then the \meta{false code} is inserted.
%For example
%\begin{demohigh}
%\IgnoreSpacesOn
%\dimSet\lTmpaDim{5pt}
%\dimCaseTF{2\lTmpaDim}{
%  {5pt}     {\prgReturn{Small}}
%  {4pt+6pt} {\prgReturn{Medium}}
%  {-10pt}   {\prgReturn{Negative}}
%}{
%  \prgReturn{[Some Match]}
%}{
%  \prgReturn{[No Match]}
%}
%\IgnoreSpacesOff
%\end{demohigh}
\end{function}

\end{document}
