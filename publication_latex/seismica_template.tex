%%%%%%%%%%%%%%%%%%%%%%%%%%%%%%%%%%%%%%%%%
% Seismica
% LuaLaTeX Template
% Version 1 (2022)
%
%
% Important note:
% This template must be compiled with LuaLaTeX, the below lines will ensure this
%!TEX TS-program = lualatex
%!TEX encoding = UTF-8 Unicode
%%%%%%%%%%%%%%%%%%%%%%%%%%%%%%%%%%%%%%%%%


% Options available: report, breakmath, proof, onecolumn, invited, opinion
% breakmath breaks the equations that are too long
% proof prints a proof watermark
% onecolumn only if issue with breakmath
% invited prints "invited" in the banner and headers

\documentclass[opinion,breakmath]{seismica}
%\documentclass[breakmath]{seismica}

%% If report, specify the type to appear in header (comment or uncomment). if not specified, will only show 'report'
%\reporttype{Fast Report}
%\reporttype{Null Results Report}
%\reporttype{Software Report}
%\reporttype{Data Report}

% Metadata by SCE team only
\dois{10.26443/seismica.v2i1.N}
\receiveddate{September 23, 2022}
\accepteddate{January 6, 2023}
\publisheddate{January 11, 2023}
\theyear{2023}
\thevolume{2}
\thenumber{1}  %% issue number
\prodedname{Efirstname Editorname}  %% Production editor
\handedname{Efirstname Editorname}  %% handling editor
\copyedname{Cfirstname Copyedname}  %% copyed editor
%\translatorname{translator name, if there is one}
%\reviewername{Rfirstname Reviewname\\ Rfirstname Reviewname}  % if at least one non-anonymous

%%%%%%%%%%%%%%%%%%%%%%%%%%%%%%%%%%%%%%%%%
% Metadata by authors (Replace everything below)

\title{A template for Seismica}
\shorttitle{A template for Seismica} % used for header

% use ~ for non-breaking spaces in authors, so names don't break across lines
\author[1]{A. Author1
	\thanks{Corresponding author: bla@som.ac.edu}
	\orcid{0000-0002-1825-0097}}
\author[1]{B.~author2 
	\orcid{0000-0002-1825-097}}
\author[2]{C.~Author3
	\orcid{0000-0002-1825-007}}
\affil[1]{affil Author 1 and 2 }
\affil[2]{affil author 3}

\credit{Funding acquisition}{Alice, Bob}
\credit{Writing}{Charlie, Doris}
\credit{Writing - review \& editing}{Emilio, Francis}

\setotherlanguages{french}
%% Do not put arabic in \setotherlanguages{} as it is not supported by polyglossia
%% Instead, use these commands within the text:
%%\begin{Arabic} and \end{Arabic} around paragraphs in Arabic
%%\n{} to wrap any digits within Arabic text that should read left-to-right
%%\textarabic{} for Arabic text embedded in a left-to-right paragraph

\begin{document}

	% up to 3 abstracts to include in {}
	% English asbtract is first
	\makeseistitle
	{%
	\begin{summary}{Abstract}
Replace text
\vspace{1.5cm} % because abstract is too short
	\end{summary}
	\begin{summary}{Non-technical summary}
	The text goes here. Again, no longer than 200 words.
	\vspace{1.5cm} % because abstract is too short
\end{summary}
 	}

 %%%% In case there are too many abstracts and need a page break, use this:
%\addsummaries{
%		\begin{summary}{Additional abstract} 
%			The text goes here. Again, no longer than 200 words.
%		\end{summary}
%		\begin{summary}{Additional abstract} 
%		The text goes here. Again, no longer than 200 words.
%	\end{summary}
%}  %% don't forget this one!
	
	\section{Introduction}
	
	Cite with \citep{metropolis_monte_1949} or \citet{metropolis_monte_1949}
	
	To refer to a figure, use Fig.~\ref{fig:2} or Figs~\ref{fig:1}, \ref{fig:2} (Tab. and Tabs).
	
	\begin{figure}[ht!]
		\includegraphics[width=\columnwidth]{empty} 
		\caption{column-wide figure.}
		\label{fig:1}
	\end{figure}
	
	\section{Section 1}
	
	\subsection{Subsection }
	
	\begin{figure*}[ht!]
		\includegraphics[width=\textwidth]{empty} 
		\caption{Full-width figure.}
		\label{fig:2}
	\end{figure*}

Refer to equation \ref{eq1}.

\begin{equation} \label{eq1}
\mathbf{G} = \frac{1}{2}(2\cos z) + (1/2)(2\cos z+j\sin z-j\sin z) + (1/2)(\cos z+j\sin z+\cos z-j\sin z) -  (1/2)(e^{jz}+e^{-jz})
\end{equation}
	
\begin{table}
	\begin{tabular}{lll}
		\thickhline
		Animal    & Description & Price (\$) \\
		\hline\rule{0pt}{2ex}
		Gnat      & per gram    & 13.65      \\
		& each        & 0.01       \\
		Gnu       & stuffed     & 92.50      \\
		Emu       & stuffed     & 33.33      \\
		Armadillo & frozen      & 8.99       \\
		\thickhline
	\end{tabular}
	\caption{In tables, you can use \code{\textbackslash hline} or \code{\textbackslash thickhline} for horizontal rules. Also, you can add a \code{\textbackslash rule{0pt}{2ex}} after \code{hlines} to add an extra vertical space.}
\end{table}

	\subsection{Code}

Code examples should be concise and descriptive. They should introduce core functionality or specific syntax and should be included using the \code{lstlisting} environment. Note that lines longer than 45 characters will be broken when using the prepress option. Extended examples or use cases should be uploaded separately. Individual words of code can be written inline, for example:

To improve stability of the inversion, the \code{Model} object accepts the \code{strict} keyword, which disables piecewise linear approximation of the target function (Listing~\ref{code}).

\begin{lstlisting}[caption=Example use of \code{Model}, label=code, language=Python]
#2 4 6 8 0 2 4 6 8 0 2 4 6 8 0 2 4 6 8 0 2 4|
import mymodule as mm

model = mm.Model(strict=True)
mdls = model.perturb()

for mdl in mdls:
	var = mdl.get_variance()
\end{lstlisting}

\section{Page formatting}
If you need to add a landscape-format page, for a table or figure, there are header/footer styles in the template. Adding a landscape page will mess with tex's page-filling algorithm since it will insert the rotated page exactly where it appears in the text, so you will need to manually move text before/after to maximize page fill.

%% add these lines (uncommented) before the landscape content
%\onecolumn
%\thispagestyle{lscapedplain}
%\pagestyle{lscapedplain}
%\begin{landscape}

%% [add your landscape content here - if a figure*, use height=\textwidth (or .9\textwidth) instead of width to set size.

%% add these lines (uncommented) after the landscape content to reset to portrait mode
%\end{landscape}
%\restoregeometry
%\thispagestyle{plainReset}
%\pagestyle{plainReset}

% If you want a caption for a full-page landscape figure to be on the next page but still have the same number, you can use
%\addcounter{figure}{-1}
% before adding another figure environment with just the caption following the landscape page. Note that this will not play well with the XML conversion.


\begin{acknowledgements}
	Thank all relevant parties and acknowledge funding sources, if any.
\end{acknowledgements}
	
\section*{Data availability}
Authors should direct readers to an open access repository such as figshare or Github, where data are made available.

\section*{Competing interests}
The authors declare no competing interests.

\bibliography{biblio} 
	
\end{document}

